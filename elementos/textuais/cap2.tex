% ----------------------------------------------------------
% PARTE
% ----------------------------------------------------------
\part{Referenciais teóricos}
% ----------------------------------------------------------

% ---
% Capitulo de revisão de literatura
% ---
\chapter{Referenciais Teóricos}
% ---

% ---
\section{O Bioma Pantanal}

	Este capítulo visa apresentar princípios mais relevantes  com a finalidade de explicar o que é o Bioma Pantanal do Dicionário Ambiental.((0))eco,conceituando o significado de Bioma no ponto de vista de alguns pesquisadores citados no artigo de \citeonline []{coutinho2006conceito}.
	
	O Pantanal é um bioma que possui uma ampla área vegetacional, que por motivos biológicos encontram-se nele características de outros biomas, como o Cerrado, a Amazônia, Mata Atlântica e o Chaco Boliviano, tornando-se fontes de estudos e pesquisas.
	
	Situado em grande parte do centro-oeste brasileiro, estendem-se pelo Paraguai, Argentina e Bolívia. Segundo o IBGE (Instituto Brasileiro de Geografia e Estatística), no Centro Oeste do Brasil é ocupado  cerca  de 189 mil quilômetros quadrados, dos quais 63\% na região de Mato Grosso do Sul e 37\% em Mato Grosso.
	
	A análise do clima é quente no verão e com temperatura média em torno de $32\,^{\circ}\mathrm{C}$, sendo frio e seco no inverno com média em torno de $21\,^{\circ}\mathrm{C}$, ocorrendo raramente geada nos meses de julho e agosto (Portal do Pantanal, 2019).	

	\subsection{Características}
	
		
	
	\subsection{Fitogeografia do Pantanal} 
	
		Este capítulo visa apresentar o estudo da distribuição geográfica por meio de pesquisas realizadas por \citeonline {ab2006brasil}, \citeonline {pott2009vegetaccao} e dos fatores históricos e biológicos que a determinaram o surgimento do Pantanal.
		
		\subsubsection{O Pantanal Mato-Grossense e a Teoria de Refúgios e Redutos}
	
			As informações obtidas referente a formação do Pantanal, demorou milhares de anos para serem coletadas, passou por processos de pesquisas e análises para obter  uma resposta que esclareça sobre a origem e a evolução do Pantanal.
	
			O espaço fisiográfico do Pantanal, iniciou-se devido a uma reativação tectônica, no qual causou o rebaixamento das planícies, causando um preenchimento detrítico de uma bacia de sedimentação,  ao longo do tempo, com o desenvolvimento e adaptação da vasta abóbada regional e de terrenos antigos, consequências da erosão, originou-se a depressão pantaneira \cite{ab2006brasil}.
	        
	
		\subsubsection{Primeiros Pesquisadores}
	
			\begin{citacao}
				“As primeiras informações sobre vegetação do Pantanal foram de viagens exploratórias de alguns dos botânicos europeus pioneiros que começaram a desvendar a flora brasileira entre 1825 e 1895, por exemplo, Langsdorff, Riedel, Weddell, Kuntze, Lindman, Malme e Spencer Moore”\cite{pott2009vegetaccao}
			\end{citacao}
		
		
			O Pantanal por possuir diversas fontes de pesquisas, despertou o interesse em alguns pesquisadores europeus.
		
			De acordo com Salis, Suzana Maria et al. (2006) a Flora Pantaneira teve suas primeiras informações registrada no final do século XIX por naturalistas europeus, como o Moore (1895) e Malme (1905), que trabalharam com espécies de algumas famílias (Leguminosae e Vochysiaceae), Veloso (1947) que classificou alguns tipos de vegetações relacionando-as com regiões inundadas e a sua composição de espécies, o levantamento e listagens de algumas espécies ocorrentes no Pantanal foram realizadas por Pott et al. (1986, 1997) e Guarim Neto, a composição florística dos capões localizados no Região do Abobral, no município de Corumbá, estado de Mato Grosso do Sul, foram estudadas por Damasceno Júnior et al (1999).
		
			Prance \& Schaller (1982), Ratter et al. (1988), Nascimento \& Cunha (1989) e Cunha (1990) realizaram os trabalhos pioneiros referente aos aspectos fitossociológicos nas regiões de florestas e cerradões do Pantanal. Os estudos das estruturas em cerradões e florestas no Pantanal da Nhecolândia no estado de Mato Grosso do Sul foram realizados por Soares (1997), Salis et al. (1999) e Salis (2000).
		
			Estudos realizados por Salis, Suzana Maria et al. (2006) sobre o Cerrado brasileiro, baseando-se no trabalho de Ratter et al. (1996, 2003) e Castro et al. (1999), com o intuito de analisar os aspectos e distribuições das vegetações encontradas no Pantanal, diante de observações notou-se um padrão geográfico na distribuição da flora, no qual seis grupos florísticos foram reconhecidos após a comparação e análise de 376 áreas do Cerrado, o grupo do Sudeste, (cerrados do Estado de São Paulo, Paraná e Sul de Minas Gerai); Centro-Sudoeste (cerrados do Distrito Federal e de Minas Gerais); Norte-Nordeste (Bahia, Ceará, Norte de Minas Gerais, Maranhão, Piauí e parte do Leste de Tocantins); Centro-Oeste (Mato Grosso, Mato Grosso do Sul, Goiás, parte do Oeste de Tocantins e Sul do Paraná); mesotrófico-oeste(cerrados de vários estados, principalmente de Rondônia e Mato Grosso do Sul) e um grupo de áreas disjuntas, formado pelas savanas amazônicas.
		
	
\section{Vegetações }

	\subsection{Famílias}
	
		\subsubsection{Espécies}
		
			
		


\section{Sistemas Operacionais}
		
	De acordo com \citeonline []{toledo2016desenvolvimento}, "São chamados smartphones os telefones celulares que oferecem alta capacidade de processamento", ou seja, oferece recursos avançados e segundo \citeonline []{torres2018biblia}, é um computador que pode ser transportado a palma da mão, e como todo computador os smartphones necessitam de sistemas operacionais para funcionar.
		
	O sistema operacional é responsável por gerenciar processos e realizar a comunicação entre aplicação e hardware, além de disponibilizar ao usuário uma interface no computador, de forma que a utilização fique mais agradável. \citeonline {velloso2014informatica}
			
	Hoje em dia o mercado de dispositivos móveis oferece numerosos modelos de aparelhos, e cada fabricante opta por um determinado sistema operacional. Para este trabalho foram selecionados dois sistemas operacionais mais utilizados no mundo segundo IDC (\textbf{ver como se referencia}), o sistema operacional Android da empresa Google e o iOS da Apple.
		
	\subsection{Android}
			
		Liderado pela Google, o Android é uma Linguagem Open Source, ou seja, de código livre para dispositivos móveis, como é um projeto Open Source a documentação é pública e livre para todos, o objetivo é evitar qualquer falha no sistema através de feedbacks transmitidos por seus usuários. Dessa maneira o Android torna-se um sistema operacional completo com código-fonte personalizável que pode ser enviado para qualquer dispositivo.
				
		%https://source.android.com/
				
	\subsection{IOS}
			
				
			
\section{Desenvolvimento Mobile}

	Este capítulo tem por objetivo apresentar as categorias de desenvolvimento mobile segundo os pesquisadores \citeonline {da2014paradigmas}, \citeonline{emdesafios} e \citeonline{toledo2016desenvolvimento}.

	\begin{citacao}
		“A evolução da tecnologia dos aparelhos celulares permitiu oferecer ao usuário recursos que vão muito além da realização de uma chamada ou do envio de uma mensagem. As melhorias de hardware dos aparelhos celulares permitiram o desenvolvimento de sistemas operacionais mais avançados.”\cite[]{da2014paradigmas}
	\end{citacao}
	
	Com o avanço dos Sistemas operacionais tornou-se factível produzir aplicativos melhores, tornando os aparelhos celulares mais comuns no cotidiano das pessoas devido a comodidade em facilitar a realização de tarefas diárias.
	
	Segundo \citeonline []{emdesafios}, os aplicativos móveis possibilitaram a criação de novos modelos de negócio, em que o usuário pode ter acesso a novos recursos e serviços, por meio de lojas virtuais produzidas por fabricantes de dispositivos móveis. O principal benefício de surgimento de novos modelos é a ligação de serviços e aplicações avançadas de Internet ao aparelho mobile. Entretanto, existem diferentes plataformas tecnológicas para desenvolvimento mobile, incluindo sistemas operacionais e IDE's  (Integrated Development Environment), gerando uma ampla diversidade de soluções disponíveis no mercado. Os aplicativos móveis dependem de uma organização de código específico para que possam ser continuamente executados em diversas plataformas e versões.
	
	\begin{citacao}
		"Limitações de plataforma para distribuição do aplicativo, como, tempo, custo para o desenvolvimento e complexidade das tecnologias necessárias para a sua criação e manutenção são pontos problemáticos em um projeto voltado ao desenvolvimento deste tipo de aplicativo."\cite{emdesafios}
	\end{citacao}
	
	
	Seguindo essa linha de pensamento existem duas maneiras de categorizar aplicativos, nativas e web, ambos com seus prós e contras. Há também aplicações multiplataformas que englobam algumas funcionalidades de ambas categorias. Pensando nisso é fundamental saber qual utilizar na hora de planejar um desenvolvimento mobile.
	
	\subsection{Desenvolvimento de Aplicações Nativas}
	
		Aplicativos Nativos são aplicações desenvolvidas para determinados SOs, com Linguagens de programação e IDEs específicas. Cada qual com sua responsabilidade, O sistema operacional é responsável por gerenciar os recursos do dispositivo, a linguagem é utilizada na programação do software e o IDE dispõe ferramentas que ajudam na programação da aplicação.
		
		Com a finalidade de melhorar a interação com o usuário, aplicações nativas utilizam recursos presentes no aparelho,como dispositivos de hardware ()GPS, microfone, magnetômetro e câmera) e software (e-mails,  calendário, contatos, fotografias e telefone), para isto os aplicativos são instalados diretamente no sistema operacional, facilitando o uso da aplicação no modo off-line, caso não haja acesso a internet.\cite{toledo2016desenvolvimento}
		
		Apesar das múltiplas funcionalidades, aplicativos nativos não são fáceis de desenvolver, pois para ser executado em diversos dispositivos é preciso utilizar uma variedade de plataformas e programar de forma que funcione diversas versões. Devido a grande quantidade de plataformas, mantê-lo atualizado é desvantajoso, pois faz-se necessário uma variedade de teste e distribuição para cada versão.
	
	\subsection{Desenvolvimento de Aplicações Web}
	
		Uma aplicação web móvel nada mais é que um site em formato para celulares, que pode ser acessado de um navegador Web instalado no dispositivo. Tal qual um ambiente web tradicional, é desenvolvido utilizando as linguagens HTML para texto estáticos e imagens, CSS para o design da tela e JavaScript que é responsável pelos efeitos na aplicação. Por serem baseados em navegadores web, as aplicações podem ser excutadas em quaisquer dispositivos que possuem acesso a internet.\cite{emdesafios}
		
		Segundo \citeonline {emdesafios}, as aplicações web são mais baratas, de fácil manutenção e não ocupam memória no dispositivo,pois, não precisam ser instaladas, o que as torna vantajosas devido ao fato de serem compatíveis com diversas plataformas.
		
		Mesmo com as diversas vantagens, é necessário o acesso constante à internet em uma velocidade satisfatória para que haja uma rápida troca de dados com os servidores, além de que se o sistema precisar utilizar recursos e funcionalidades do dispositivo as aplicações web podem ser consideradas descartadas.\cite{toledo2016desenvolvimento}.
		
		Como as aplicações nativas, aplicativos web necessitam de vários testes para adequar ao tamanho de tela dos dispositivos e as exigências de cada navegador, uma vez que o mercado de smartphones é amplo e cada fabricante faz uso de navegadores que se adaptam melhor ao seus produtos.
			
	\subsection{Desenvolvimento de Aplicações Híbridas}
	
		As  aplicações híbridas é a junção de aplicativos nativos e WEB, pois como os nativos, aplicações híbridas são baixadas diretamente no dispositivo e fazer uso de funcionalidades e recursos do mesmo, por exemplo a câmera, bluetooth e GPS. Assim com aplicativos WEB, os híbridos fazem uso de linguagens HTML, JavaScript e CSS. \cite{tavares2016introduccao}
		
		A vantagem deste tipo de aplicação é a facilidade no desenvolvimento, pois é preciso apenas um código para ser compilado para diversas plataformas, poupando tempo para o  desenvolvedor, dado que não se faz necessário a programação em várias linguagens.
	
\section{Frameworks}


	\subsection{Laravel}
	
		
	
\section{Ambientes de Desenvolvimento}
	
	\subsection{Visual Studio Code}
	
	IDEs: vscode, mysql workbench
	
	Linguagens: PHP, Javascript, HTML5, CSS3
	
	Frameworks: Laravel e React Native

	\subsection{composer}
	
	\subsection{node e npm}
	
		\subsection{JavaScript}
	
	\subsection{React Native} %citar 
	\subsection{node Js} 
 









