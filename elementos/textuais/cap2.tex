% ----------------------------------------------------------
% PARTE
% ----------------------------------------------------------
\part{Referenciais teóricos}
% ----------------------------------------------------------

% ---
% Capitulo de revisão de literatura
% ---
\chapter{Referenciais Teóricos}
% ---

% ---
\section{O Bioma Pantanal}

	\subsection{Características}
	
		Este tópico visa conceitualizar características do Bioma Pantanal, tais como o conceito da palavra Bioma, apresentar o comportamento climático do Pantanal juntamente com sua área ocupacional e sua demais caraterísticas.
		 
		O termo Bioma, segundo Colinvaux (1993), foi proposto por Shelford, que do grego, Bio significa vida e  Oma, significa massa. No ponto de vista de Font Quer (1953), a mesma definição foi criado pelo pesquisador Clements. O diferencial entre um autor e outro, está relacionado com caracterização da palavra formação e da palavra bioma, visto que antes a formação referia-se apenas à vegetação, e bioma ao conjunto de vegetação com associação da fauna. Agora, nesse novo termo de Font Quer, foi totalmente incluso a fauna. Já no glossário do livro de Clements (1949), encontra-se a definição para bioma: “\textit{Biome- A community of plants and animals, usually of the rank a formation: a biotic community}”. 
		
		O Pantanal é um dos seis biomas brasileiros, considerado um dos mais deslumbrantes e menos afetado pelos impactos ambientais em relação aos demais, o qual ocupa cerca de 25\% da região de Mato Grosso do Sul e 7\% em Mato Grosso, totalizando em aproximadamente 150 mil quilômetros quadrados no Brasil (IBGE,2004), possuindo uma ampla área vegetacional, que por motivos biológicos, encontram-se nele características de outros biomas, como o Cerrado, a Amazônia, a Mata Atlântica e o Chaco boliviano. No verão o clima é quente e com temperatura média em torno de $32\,^{\circ}\mathrm{C}$, no inverno é frio e seco, com média em torno de $21\,^{\circ}\mathrm{C}$ \cite{portal}. A temperatura, o clima e o solo  acabam influenciando no desenvolvimento de algumas espécies de plantas que predominam no Bioma, havendo distribuição da mesma, o qual buscam o ambiente que mais favorece seu desenvolvimento.
		
		\begin{citacao}
			O Pantanal mato-grossense é constituído por um complexo ou mosaico de diferentes biomas florestais de hidro biomas e helô-biomas (carandazais, paratudais), savânicos de piro-peino biomas (cerrados das cordilheiras entre lagoas), florestais de lito biomas (florestas tropicais estacionais caducifólios sobre afloramentos rochosos e solos rasos), campestres de hidro-helô-biomas (campos inundáveis), em meio a rios, lagoas de água doce (baias), lagoas de água salobra e alcalina (salinas) etc., todos pertencentes ao Zonobioma II. \cite{coutinho2006conceito})
		\end{citacao}
		 
		
		A região pantaneira apresenta uma vegetação de fácil identificação, ou seja, não é homogênea, tendo a flora diferenciada, de acordo com o solo e a altitude. Podemos analisar  a vegetação do Cerrado, que apresenta árvores de porte médio, entre arbustos e gramíneas. Os capões de mato são encontrados acima das áreas inundáveis e, consequentemente, a paisagem do Pantanal sofre periodicamente mudanças, em épocas de cheia, o rio transborda, alagando campos e morros isolados, formando pequenas ilhas que servem como abrigo para os animais que foram atingidos. Na seca, os rios retomam seus leitos naturais,  revelando os campos conforme a água vai baixando. \cite{eco}
		
	\subsection{Fitogeografia do Pantanal} 
	
		Existem várias teorias que explicam o surgimento do Bioma Pantanal,  este tópico visa apresentar o estudo da distribuição geográfica, dos fatores históricos e biológicos que determinaram o surgimento do Pantanal, por meio de pesquisas realizadas por \citeonline{ab2006brasil} e \citeonline{pott2009vegetaccao}.
		
		Várias teorias surgiram no decorrer do tempo para explicar o surgimento do Pantanal. De acordo com \citeonline[p. ]{ab2006brasil}, geólogo que desenvolveu estudos sobre o espaço brasileiro, as informações obtidas referente a formação do Pantanal demoraram milhares de anos para serem coletadas, passou por processos de pesquisas e análises para obter  uma resposta que esclareça a origem e a evolução do Pantanal.
		
		Uma das teorias, a defendida por \apudonline{ruellan1952escudo}{ab2006brasil}, referência-se a Depressão  do Pantanal como um exemplo de \textit{boutonnière}, uma linguagem geomorfológica francesa, que tem como significado “casa de botão”, o termo refere-se ao Bioma como a casa e as espécies de plantas,os botões. O qual busca identificar muitas estruturas das vegetações que foram modificadas e escavadas ao longo do tempo em terreno Pré-Cambriano, que, por consequências dos processos erosivos, causaram esvaziamento do terreno. Conforme abordado na teoria dos Refúgios e Redutos, o espaço fisiográfico do Pantanal, iniciou-se devido a uma reativação tectônica, o qual causou o rebaixamento das planícies causando um preenchimento detrítico de uma bacia de sedimentação. Ao longo do tempo, por consequências da erosão, originou-se a depressão pantaneira, com o desenvolvimento e adaptação da vasta abóbada regional e de terrenos atingidos \cite{ab2006brasil}. Antes do surgimento da denominação utilizada nos dias atuais, Bioma Pantanal, era considerado apenas como uma vasta abóbada(amontoado de pedras,morros) de escudo, que servia como fornecimento detrítico para as bacias do Grupo Bauru e Parecis.
		
		Na atualidade, o pantanal constitui uma extensa planície de acumulação, com topografia plana e alagada periodicamente, tendo o rio Paraguai e seus afluentes como o principal meio de transporte de água e sedimentos. \cite{souza2006origem}
		
		Segundo pesquisas realizadas por \cite{de2010pantanal}, o Pantanal é considerado a maior planície alagada contínua do mundo, possuindo em território brasileiro cerca de 140.000 $km^{2}$,  situado nos estados de Mato Grosso e Mato Grosso do Sul.
		
		O Pantanal tem a vegetação como um dos seus principais fatores atrativos, possuindo cerca de 2.000 espécies de fanerógamas, incluindo 200 exóticas, tendo como as principais famílias: leguminosas e gramíneas. As terrestres herbáceas, aproximadamente 1000 espécies (Pott, 2009).
		
		\begin{citacao}
			A origem da flora do Pantanal vinha sendo atribuída à influência de Cerrado, Amazônia, Mata Atlântica e Chaco, sem o devido levantamento de espécies. Com base em coletas botânicas para herbário e listagem florística, as proporções fitogeográficas são 50\% de espécies de ampla distribuição, 30\% de espécies do Cerrado, e 20\% de outras origens. \cite{pott2009vegetaccao}
		\end{citacao}
				
		O Bioma Pantanal possui uma ampla área vegetacional, várias espécies de plantas são encontradas na região, porém, não havia um levantamento de informações para  identificar  quais biomas pertenciam, visto que as mesmas podem ser encontradas em mais de um bioma, tornando-se assim fontes de estudos e pesquisas.		
		
		Os diversos biomas que compõem o Bioma Pantanal estão distribuídos de forma que seja possível obter sua identificação. O Chaco boliviano se localiza ao sudeste do Pantanal, nas sub-regiões de Porto Murtinho e Nabileque, no Estado de Mato Grosso do Sul. O Cerrado predomina na parte leste e a vegetação com características da Amazônia são encontradas juntos aos rios, em partes baixas situado ao Oeste.
		
		\subsubsection{Primeiros Pesquisadores}
		
			O Pantanal por possuir uma rica flora, acabou despertando interesses em alguns pesquisadores europeus, que foram os primeiros a estudarem a vegetação pantaneira. Estudos que demoraram milhares de anos, passando por pesquisas e análises profundas para poder esclarecer a origem e evolução do Bioma.
			
			\begin{citacao}
				As primeiras informações sobre vegetação do Pantanal foram de viagens exploratórias de alguns dos botânicos europeus pioneiros que começaram a desvendar a flora brasileira entre 1825 e 1895. \cite{pott2009vegetaccao}
			\end{citacao}
			
			De acordo com \citeonline{salis2006distribuiccao}, a Flora Pantaneira teve suas primeiras informações registrada no final do século XIX por naturalistas europeus, como o \apudonline{le1895phanerogamic}{salis2006distribuiccao} e \apudonline{malme1905vochysiaceen}{salis2006distribuiccao}, que trabalharam com espécies de algumas famílias (Leguminosae e Vochysiaceae), \apudonline{veloso1947consideraccoes}{salis2006distribuiccao} que classificou alguns tipos de vegetações relacionando-as com regiões inundadas e a sua composição de espécies. O levantamento e listagens de algumas espécies ocorrentes no Pantanal foram realizadas por \apudonline{pott1986flora}{salis2006distribuiccao} e \apudonline{guarim1984plantas}{salis2006distribuiccao}, a composição florística dos capões localizados no Região do Abobral, no município de Corumbá, Estado de Mato Grosso do Sul, foram estudadas por \apudonline{damasceno1996aspectos}{salis2006distribuiccao}. \apudonline{prance1982preliminary}{salis2006distribuiccao}, \apudonline{ratter1988observations}{salis2006distribuiccao}, \apudonline{nascimento1989estrutura}{salis2006distribuiccao} e \apudonline{cunha1990estudo}{salis2006distribuiccao} desenvolveram trabalhos referentes aos aspectos fitossociológicos nas regiões de florestas e cerradões do Pantanal e os estudos referentes a estruturas em cerradões e florestas no Pantanal da Nhecolândia no Estado de Mato Grosso do Sul, foram realizados por \apudonline{soares1997estrutura}{salis2006distribuiccao}, Salis et al. (1999) e \apudonline{salis2000fitossociologia}{salis2006distribuiccao}.
			
			Estudos realizados por  Ratter et al. (1996, 2003) e Castro et al. (1999), sobre o Cerrado brasileiro, tiveram o intuito de levantar dados sobre os aspectos e distribuições das vegetações encontradas no Pantanal. Diante das observações notou-se um padrão geográfico na distribuição da flora, no qual seis grupos florísticos foram reconhecidos após a comparação e análise de 376 áreas do Cerrado, o grupo do Sudeste, (cerrados do Estado de São Paulo, Paraná e Sul de Minas Gerai); Centro-Sudoeste (cerrados do Distrito Federal e de Minas Gerais); Norte-Nordeste (Bahia, Ceará, Norte de Minas Gerais, Maranhão, Piauí e parte do Leste de Tocantins); Centro-Oeste (Mato Grosso, Mato Grosso do Sul, Goiás , parte do Oeste de Tocantins e Sul do Paraná ); mesotrófico-oeste(cerrados de vários estados, principalmente de Rondônia e Mato Grosso do Sul) e um grupo de áreas disjuntas, formado pelas savanas amazônicas. \cite{souza2006origem}
	
\section{Vegetações}

	Neste tópico serão abordados conteúdos sobre as oito famílias escolhidas para o desenvolvimento deste trabalho, retratando o nome científico, nome popular, época de floração, grau de ameaça, entre outros.

	\subsection{Famílias}
		
		As famílias citadas a seguir, possuem  árvores de porte médio e grande. O qual as características das plantas de cada família foram coletadas do site Web Ambiente, o qual contempla o maior banco de dados já produzido no Brasil sobre espécies vegetais nativas além de disponibilizar estratégias para recomposição ambiental.O sistema foi desenvolvido pela EMBRAPA e pela Secretaria de Extrativismo e Desenvolvimento Rural Sustentável - MMA, em cooperação com diversos especialistas de diferentes instituições parceiras.
		
		\begin{description}
			\item[Anacardiceae]
			\item[Arecaceae] 
			\item[Bignoniaceae]
			\item[Boraginaceae]
			\item[Ebenaceae]
			\item[Fabaceae]
			\item[Lamiaceae]
			\item[Rubiaceae]
			
		\end{description}
		
		\section{Desenvolvimento Mobile}
		
		Este capítulo tem por objetivo apresentar as categorias de desenvolvimento mobile segundo os pesquisadores \citeonline {da2014paradigmas}, \citeonline{emdesafios} e \citeonline{toledo2016desenvolvimento}.
		
		\begin{citacao}
			“A evolução da tecnologia dos aparelhos celulares permitiu oferecer ao usuário recursos que vão muito além da realização de uma chamada ou do envio de uma mensagem. As melhorias de hardware dos aparelhos celulares permitiram o desenvolvimento de sistemas operacionais mais avançados.”\cite[]{da2014paradigmas}
		\end{citacao}
		
		Com o avanço dos Sistemas Operacionais tornou-se factível produzir aplicativos melhores, tornando os aparelhos celulares mais comuns no cotidiano das pessoas devido a comodidade em facilitar a realização de tarefas diárias.
		
		Segundo \citeonline []{emdesafios}, os aplicativos móveis possibilitaram a criação de novos modelos de negócio, em que o usuário pode ter acesso a novos recursos e serviços, por meio de lojas virtuais produzidas por fabricantes de dispositivos móveis. O principal benefício de surgimento de novos modelos é a ligação de serviços e aplicações avançadas de Internet ao aparelho mobile. Entretanto, existem diferentes plataformas tecnológicas para desenvolvimento mobile, incluindo sistemas operacionais e IDE's  (Integrated Development Environment), gerando uma ampla diversidade de soluções disponíveis no mercado. Os aplicativos móveis dependem de uma organização de código específico para que possam ser continuamente executados em diversas plataformas e versões.
		
		\begin{citacao}
			"Limitações de plataforma para distribuição do aplicativo, como, tempo, custo para o desenvolvimento e complexidade das tecnologias necessárias para a sua criação e manutenção são pontos problemáticos em um projeto voltado ao desenvolvimento deste tipo de aplicativo."\cite{emdesafios}
		\end{citacao}
				
		Seguindo essa linha de pensamento existem duas maneiras de categorizar
		aplicativos, nativas e web, ambos com seus prós e contras. Há também aplicações multiplataformas que englobam algumas funcionalidades de ambas categorias. Pensando nisso é fundamental saber qual utilizar na hora de planejar um desenvolvimento mobile.
		
		\subsection{Desenvolvimento de Aplicações Nativas}
		
		Aplicativos Nativos são aplicações desenvolvidas para determinados SOs, com Linguagens de programação e IDEs específicas. Cada qual com sua responsabilidade, O sistema operacional é responsável por gerenciar os recursos do dispositivo, a linguagem é utilizada na programação do software e o IDE dispõe ferramentas que ajudam na programação da aplicação.
		
		Com a finalidade de melhorar a interação com o usuário, aplicações nativas utilizam recursos presentes no aparelho,como dispositivos de hardware (GPS, microfone, magnetômetro e câmera) e software (e-mails,  calendário, contatos, fotografias e telefone), para isto os aplicativos são instalados diretamente no sistema operacional, facilitando o uso da aplicação no modo off-line, caso não haja acesso a internet.\cite{toledo2016desenvolvimento}
		
		Apesar das múltiplas funcionalidades, aplicativos nativos não são fáceis de desenvolver, pois para ser executado em diversos dispositivos é preciso utilizar uma variedade de plataformas e programar de forma que funcione diversas versões. Devido a grande quantidade de plataformas, mantê-lo atualizado é desvantajoso, pois faz-se necessário uma variedade de teste e distribuição para cada versão.
		
		\subsection{Desenvolvimento de Aplicações Web}
		
		Uma aplicação web móvel nada mais é que um site em formato para celulares, que pode ser acessado de um navegador Web instalado no dispositivo. Tal qual um ambiente web tradicional, é desenvolvido utilizando as linguagens HTML para texto estáticos e imagens, CSS para o design da tela e JavaScript que é responsável pelos efeitos na aplicação. Por serem baseados em navegadores web, as aplicações podem ser executadas em quaisquer dispositivos que possuem acesso a internet.\cite{emdesafios}
		
		Segundo \citeonline {emdesafios}, as aplicações web são mais baratas, de fácil manutenção e não ocupam memória no dispositivo,pois, não precisam ser instaladas, o que as torna vantajosas devido ao fato de serem compatíveis com diversas plataformas.
		
		Mesmo com as diversas vantagens, é necessário o acesso constante à internet em uma velocidade satisfatória para que haja uma rápida troca de dados com os servidores, além de que se o sistema precisar utilizar recursos e funcionalidades do dispositivo as aplicações web podem ser consideradas descartadas.\cite{toledo2016desenvolvimento}.
		
		Como as aplicações nativas, aplicativos web necessitam de vários testes para adequar ao tamanho de tela dos dispositivos e as exigências de cada navegador, uma vez que o mercado de smartphones é amplo e cada fabricante faz uso de navegadores que se adaptam melhor ao seus produtos.
		
		\subsection{Desenvolvimento de Aplicações Híbridas}
		
		As  aplicações híbridas é a junção de aplicativos nativos e WEB, pois como os nativos, aplicações híbridas são baixadas diretamente no dispositivo e fazer uso de funcionalidades e recursos do mesmo, por exemplo a câmera, bluetooth e GPS. Assim com aplicativos WEB, os híbridos fazem uso de linguagens HTML, JavaScript e CSS. \cite{tavares2016introduccao}
		
		A vantagem deste tipo de aplicação é a facilidade no desenvolvimento, pois é preciso apenas um código para ser compilado para diversas plataformas, poupando tempo para o  desenvolvedor, dado que não se faz necessário a programação em várias linguagens.
		
\section{Ambientes de Desenvolvimento}
Neste tópico serão apresentadas as plataformas IDEs que serão utilizadas neste trabalho.

	\subsection{Visual Studio Code}

		O Visual Studio Code, também conhecido como VScode, é uma IDE leve, porém poderoso. Esta possui extensões para diversas linguagens, como PHP e HTML, e dispõe de suporte para JavaScript, TypeScript e Node.js. A IDE possui variadas funcionalidades que auxiliam na produtividade durante o desenvolvimento, por exemplo o conjunto de atalhos de teclado que agiliza o processo de edição do código fonte.\cite{vscode}

	\subsection{MySql Workbench}

	O MySQL Workbench trata-se de uma ferramenta visual unificada para arquitetos de banco de dados, o qual encontra-se disponível para Windows, Linux e Mac OS X. Por meio dele é possível modelar, gerar e gerenciar visualmente banco de dados, pois a ferramenta fornece modelagem de dados de SQL e ferramentas de administração para configuração de servidor, administração de usuários, backup, entre outros paradigmas.


\section{Sistemas Operacionais}

	No presente tópico será abordado os conceitos de Sistemas Operacionais, tais como suas funções tecnológicas.
			
	De acordo com \citeonline{toledo2016desenvolvimento}, "São chamados smartphones os telefones celulares que oferecem alta capacidade de processamento", ou seja, oferece recursos avançados e segundo \citeonline []{torres2018biblia}, é um computador que pode ser transportado a palma da mão, e como todo computador os smartphones necessitam de sistemas operacionais para que possam funcionar.
		
	O sistema operacional é responsável por gerenciar processos e realizar a comunicação entre aplicação e hardware, além de disponibilizar ao usuário uma interface, de forma que a utilização fique mais agradável. \cite {velloso2014informatica}
			
	Hoje em dia o mercado de dispositivos móveis oferece numerosos modelos de aparelhos, e cada fabricante opta por um determinado sistema operacional. Para este trabalho foram selecionados dois sistemas operacionais o Android da empresa Google e o iOS da Apple. 
		
	\subsection{Android}
			
		Liderado pela Google, o Android é uma Linguagem Open Source, ou seja, de código livre para dispositivos móveis, como é um projeto Open Source a documentação é pública e livre para todos, o objetivo é evitar qualquer falha no sistema através de feedbacks transmitidos por seus usuários. Dessa maneira o Android torna-se um sistema operacional completo com código-fonte personalizável que pode ser enviado para qualquer dispositivo.
				
		%https://source.android.com/
				
	\subsection{iOS}
			
			Lançado em 2007 junto com o iPhone, o iOS apenas é executado em dispositivos da Apple como iPhone, Apple TV, iPod Touch e iPad. O desenvolvimento nativo de aplicações para iOS é permitido apenas em computadores MAC, devido ao fato de que as ferramentas de desenvolvimento são encontradas apenas nos sistemas da Apple.\cite{garbin2014sistema}
			
			De acordo com \citeonline{leite2017comparativo}, pesquisadores que realizaram um comparativo entre o SO Android e iOS, aplicações da Google são executados normalmente em iOS, o que facilita no desenvolvimento de aplicativos híbridos.
				
\section{Frameworks}
	
	De acordo com \citeonline {sawaya2002dicionario}, autora do dicionário de Informática \& Internet, framework é um “conjunto de elementos e suas interligações constituindo a base de um sistema ou projeto”.
	
	Frameworks servem como ponto de partida no desenvolvimento de um projeto, pois reduz o código-fonte reaproveitando métodos, classes e funções presentes no projeto. Além disso, os frameworks fornecem um design padrão e ajustável, para tornar a interface agradável.\cite{gabardo2017laravel}
	
	Segundo \citeonline {gabardo2017laravel}, existem diversos frameworks para quase todas as linguagens, de forma que o desenvolvedor possa escolher a que melhor atende as necessidades do sistema . Para este trabalho foram escolhidas duas Frameworks, para desenvolvimento mobile e web.
	
	\subsection{React Native}
	
		Desenvolvido pelo Facebook, o React foi criado para produzir interfaces de usuário, utilizando a linguagem JavaScript. O React divide-se em duas categorias, ambas Open Source, ReactJS e React Native. Lançada em 2013, o ReactJS é utilizado para desenvolvimento web e oferece uma rápida resposta às entradas do usuário, devido a utilização do JavaScript \cite{vilete2018frameworks}.
		
		De acordo com \citeonline{vilete2018frameworks}, o React Native possui estrutura de desenvolvimento híbrido para aplicações móveis. Lançado em 2015, o framework utiliza recursos nativos de ambos, pois ao ser executado compila o código-fonte paras as linguagens Java e Swift, linguagens de programação das plataformas Android e iOS respectivamente.
		
	\subsection{Laravel}
	
	O framework Laravel foi lançado em 2011, foi desenvolvido com o intuito de agilizar a criação de aplicações web. É um framework, que atualmente vem sendo cada vez mais utilizado pelos os programadores, pois um dos principais fatores que o tornam “especial”, é por possuir uma fácil instalação e configuração. 
	
	\begin{citacao}
		Verificou-se os tipos especiais de teste que o framework Laravel fornece, como a parametrização de dados que, além de facilitar o desenvolvimento de software em relação aos relacionamentos do banco de dados, facilita o processo de automação de teste de software.\cite{pelizza2018estudo}
	\end{citacao}
	
	
	O Laravel é baseado na linguagem PHP, e uma de suas desvantagens é a necessidade de requisitos/extensões para um bom funcionamento. Porém, o Laravel possui uma arquitetura MVC (\textit{Model-View-Controller}) e tem como aspecto principal, o de auxiliar no desenvolvimento de aplicações seguras e de alto desempenho de forma ágil e simplificada, com código limpo.\cite{pelizza2018estudo}
		
\section{Linguagens de Programação}	
	
	\subsection{PHP}
		
		\subsubsection{Composer}
		
			O composer trata-se de uma ferramenta para gerenciamento de dependências em PHP, permitindo que o desenvolvedor declare as bibliotecas das quais o projeto é dependente, para que assim seja realizado o gerenciamento(instalar/atualizar). \cite{composer}
			
				
	\subsection{JavaScript}
	
		A linguagem JavaScript foi desenvolvida com o objetivo de fornecer interatividade a uma página web. A primeira versão foi lançada em 1995 e implementada em março de 1996 pela Netscape, uma empresa de serviços de computadores do EUA, em parceria com a Sun Microsystems,era uma empresa que vendia computadores, componentes de computadores e software, que contribuiu para o desenvolvimento de várias tecnologias. A San foi comprada pela Oracle Corporation em 2009,uma empresa de tecnologia multinacional americana.
		
		O básico do JavaScript é definir uma API mínima para trabalhar com textos, arrays, datas e expressões regulares, não incluindo funções de entradas e saídas, como conexão de redes, armazenamento e gráficos. É utilizada pela maioria dos sites e navegadores modernos, incluindo consoles de jogos.
		
		
	\subsection{HTML5}
		
	\subsection{CSS3}
		
		%\subsection{node e npm}
	
 









