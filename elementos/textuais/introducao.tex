% ----------------------------------------------------------
% Introdução (exemplo de capítulo sem numeração, mas presente no Sumário)
% ----------------------------------------------------------
\chapter{Introdução}
% ----------------------------------------------------------


O Pantanal é um dos seis biomas brasileiros, ainda que seja o menor é considerado  pouco afetado pelos impactos ambientais em comparação aos demais. Por possuir uma ampla flora, são encontradas inúmeros tipos de espécies diferentes, podendo estas pertencer a mais de um bioma, como o Cerrado, Amazônia, Mata Atlântica e o Chaco Boliviano.

O surgimento do Bioma Pantanal segundo o geólogo \citeonline{ab2006brasil}, ocorreu por meio de abalos tectônicos, o qual causou uma enorme erosão causando o rebaixamento das planícies e levando ao preenchimento detrítico das bacias sedimentares, que ficou denominada como Abóbada. Ao longo do tempo conforme a adaptação e desenvolvimento da vasta Abóbada originou-se o Bioma Pantanal, denominado na época Depressão Pantaneira. 

O Bioma Pantanal é um importante fator a ser estudado, podendo ser fonte de pesquisas acadêmicas e científicas. Porém, são poucos os meios em que são disponibilizados  conteúdos relacionados a flora e fauna do Bioma Pantanal. O presente trabalho tem como objetivo abranger temáticas referente a flora Pantaneira, dos quais foram selecionadas 20 plantas encontradas nesta região, as mesmas foram retiradas do site Web Ambiente e Flora do Brasil 2020, ambas possuem um dos maiores bancos de dados que fornecem informações sobre a flora de todos os biomas brasileiros. 

Tendo definido o objetivo do trabalho,o meio de disponibilizar conteúdos das espécies elegidas, será mediante ao desenvolvimento de uma aplicação mobile híbrida, que seguirá o modelo de um catálogo, o qual disponibilizará informações relevantes sobre a planta, como nome científico, nome popular, área ocupacional, grau de ameaça  e distribuição.

A aplicação tem seu desenvolvimento baseado em uma nova tecnologia, denominada React Native, um framework mobile baseado no React desenvolvido pela empresa Facebook. Este permite o desenvolvimento mobile híbrido, de forma que possa ser utilizado em dispositivos com Sistema Operacional Android e iOS.


%O Pantanal faz parte  um dos seis biomas continentais brasileiros e segundo Munick Suçuarana, possui uma área com cerca de 250 mil $Km^{2}$, no qual abrange três países: Bolívia, Paraguai e Brasil, em que aproximadamente 62\% se encontra neste último. Localizando-se nos estados de Mato Grosso e Mato Grosso do Sul, tem como parte da sua hidrografia os rios Apa, Cuiabá, Miranda, Paraguai e seus diversos afluentes que formam amplas áreas alagadiças.

%A flora pantaneira é formada por uma variedade de plantas dos biomas: Cerrado, Amazônia e Mata Atlântica, e contém poucas espécies nativas do Pantanal.Segundo a EMBRAPA, devido a sua flora diversificada, as plantas se organizam de forma fitogeográfica com as características específicas do bioma de origem.

%Pott\&Pott diz que há pouca documentação sobre a fitogeografia da flora pantaneira e surge a necessidade de pesquisas sobre a vegetação. Pensando nisso este trabalho tem por finalidade facilitar a procura de informações a respeito do complexo pantaneiro, realizando a catalogação das plantas e disponibilizando os dados em uma aplicação mobile.

%Para o desenvolvimento da aplicação será utilizado a tecnologia React Native.

