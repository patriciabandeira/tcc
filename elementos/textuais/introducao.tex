% ----------------------------------------------------------
% Introdução (exemplo de capítulo sem numeração, mas presente no Sumário)
% ----------------------------------------------------------
\chapter{Introdução}
% ----------------------------------------------------------




O Pantanal faz parte  um dos seis biomas continentais brasileiros e segundo Munick Suçuarana, possui uma área com cerca de 250 mil $Km^{2}$, no qual abrange três países: Bolívia, Paraguai e Brasil, em que aproximadamente 62\% se encontra neste último. Localizando-se nos estados de Mato Grosso e Mato Grosso do Sul, tem como parte da sua hidrografia os rios Apa, Cuiabá, Miranda, Paraguai e seus diversos afluentes que formam amplas áreas alagadiças.

A flora pantaneira é formada por uma variedade de plantas dos biomas: Cerrado, Amazônia e Mata Atlântica, e contém poucas espécies nativas do Pantanal.Segundo a EMBRAPA, devido a sua flora diversificada, as plantas se organizam de forma fitogeográfica com as características específicas do bioma de origem.

Pott\&Pott diz que há pouca documentação sobre a fitogeografia da flora pantaneira e surge a necessidade de pesquisas sobre a vegetação. Pensando nisso este trabalho tem por finalidade facilitar a procura de informações a respeito do complexo pantaneiro, realizando a catalogação das plantas e disponibilizando os dados em uma aplicação mobile.

Para o desenvolvimento da aplicação será utilizado a tecnologia React Native.

