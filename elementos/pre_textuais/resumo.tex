% ---
% RESUMOS
% ---

% resumo em português
\setlength{\absparsep}{18pt} % ajusta o espaçamento dos parágrafos do resumo
\begin{resumo}
	
	Tendo em vista que há poucas informações disponíveis sobre as diversas espécies de plantas que predominam na região pantaneira, foi pensado em um meio de dispor tais informações a fim de auxiliar em pesquisas futuras e ofertar um conhecimento a mais para a população residente na região.
	
	A disponibilização de informações ocorrerá por meio de uma aplicação mobile em forma de catálogo, fazendo o uso de uma nova tecnologia denominada React Native, pois ela permite a criação de aplicativos híbridos, ou seja, para android e IOS.
	
	Foi escolhido o desenvolvimento de uma aplicação devido a alta utilização de aparelhos celulares nos dias atuais. A aplicação irá conter informações gerais como o nome científico e popular, ocorrência, distribuição geográfica, grau de ameaça,época de floração.
	
	Espera-se, que com o desenvolvimento da aplicação, o  catálogo mobile possa contribuir com a população como uma forma de adquirir mais conhecimentos sobre a flora pantaneira e também auxiliar em pesquisas no ambiente acadêmico. 25 de maio de 2019


 \textbf{Palavras-chave}: latex. abntex. editoração de texto.
\end{resumo}

% resumo em inglês
\begin{resumo}[Abstract]
 \begin{otherlanguage*}{english}
   This is the english abstract.

   \vspace{\onelineskip}
 
   \noindent 
   \textbf{Keywords}: latex. abntex. text editoration.
 \end{otherlanguage*}
\end{resumo}