%% abtex2-modelo-trabalho-academico.tex, v-1.9.7 laurocesar
%% Copyright 2012-2018 by abnTeX2 group at http://www.abntex.net.br/ 
%%
%% This work may be distributed and/or modified under the
%% conditions of the LaTeX Project Public License, either version 1.3
%% of this license or (at your option) any later version.
%% The latest version of this license is in
%%   http://www.latex-project.org/lppl.txt
%% and version 1.3 or later is part of all distributions of LaTeX
%% version 2005/12/01 or later.
%%
%% This work has the LPPL maintenance status `maintained'.
%% 
%% The Current Maintainer of this work is the abnTeX2 team, led
%% by Lauro César Araujo. Further information are available on 
%% http://www.abntex.net.br/
%%
%% This work consists of the files abntex2-modelo-trabalho-academico.tex,
%% abntex2-modelo-include-comandos and abntex2-modelo-references.bib
%%

% ------------------------------------------------------------------------
% ------------------------------------------------------------------------
% abnTeX2: Modelo de Trabalho Academico (tese de doutorado, dissertacao de
% mestrado e trabalhos monograficos em geral) em conformidade com 
% ABNT NBR 14724:2011: Informacao e documentacao - Trabalhos academicos -
% Apresentacao
% ------------------------------------------------------------------------
% ------------------------------------------------------------------------

\documentclass[
	% -- opções da classe memoir --
	12pt,				% tamanho da fonte
%	openright,			% capítulos começam em pág ímpar (insere página vazia caso preciso)
	oneside,			% para impressão em recto e verso. Oposto a oneside
	a4paper,			% tamanho do papel. 
	% -- opções da classe abntex2 --
	%chapter=TITLE,		% títulos de capítulos convertidos em letras maiúsculas
	%section=TITLE,		% títulos de seções convertidos em letras maiúsculas
	%subsection=TITLE,	% títulos de subseções convertidos em letras maiúsculas
	%subsubsection=TITLE,% títulos de subsubseções convertidos em letras maiúsculas
	% -- o\part{title}pções do pacote babel --
	english,			% idioma adicional para hifenização
	french,				% idioma adicional para hifenização
	spanish,			% idioma adicional para hifenização
	brazil				% o último idioma é o principal do documento
	]{configuracoes/tcc}

% ---
% Pacotes básicos 
% ---
\usepackage{lmodern}			% Usa a fonte Latin Modern
%\usepackage{times}				% Usa a fonte Times
\usepackage{mathptmx}				% Usa a fonte Times
\usepackage[T1]{fontenc}		% Selecao de codigos de fonte.
\usepackage[utf8]{inputenc}		% Codificacao do documento (conversão automática dos acentos)
\usepackage{indentfirst}		% Indenta o primeiro parágrafo de cada seção.
\usepackage{color}				% Controle das cores
\usepackage{graphicx}			% Inclusão de gráficos
\usepackage{microtype} 			% para melhorias de justificação
\usepackage{varwidth}
\usepackage{array}
% ---
		
% ---
% Pacotes adicionais, usados apenas no âmbito do Modelo Canônico do abnteX2
% ---
\usepackage{lipsum}				% para geração de dummy text
% ---

% ---
% Pacotes de citações
% ---
\usepackage[brazilian,hyperpageref]{backref}	 % Paginas com as citações na bibl
\usepackage[alf]{abntex2cite}	% Citações padrão ABNT
%Pacote para justificar textos
\usepackage{ragged2e}

% --- 
% CONFIGURAÇÕES DE PACOTES
% --- 

% ---
% Configurações do pacote backref
% Usado sem a opção hyperpageref de backref
\renewcommand{\backrefpagesname}{Citado na(s) página(s):~}
% Texto padrão antes do número das páginas
\renewcommand{\backref}{}
% Define os textos da citação
\renewcommand*{\backrefalt}[4]{
	\ifcase #1 %
		Nenhuma citação no texto.%
	\or
		Citado na página #2.%
	\else
		Citado #1 vezes nas páginas #2.%
	\fi}%
% ---

% ---
% Configurações de aparência do PDF final

% alterando o aspecto da cor azul
\definecolor{blue}{RGB}{41,5,195}

% informações do PDF
\makeatletter
\hypersetup{
	%pagebackref=true,
	pdftitle={\@title}, 
	pdfauthor={\@author},
	pdfsubject={\imprimirpreambulo},
	pdfcreator={LaTeX with abnTeX2},
	pdfkeywords={abnt}{latex}{abntex}{abntex2}{trabalho acadêmico}, 
	colorlinks=true,       		% false: boxed links; true: colored links
	linkcolor=black,          	% color of internal links
	citecolor=black,        		% color of links to bibliography
	filecolor=magenta,      		% color of file links
	urlcolor=blue,
	bookmarksdepth=4
}
\makeatother
% --- 

% ---
% Posiciona figuras e tabelas no topo da página quando adicionadas sozinhas
% em um página em branco. Ver https://github.com/abntex/abntex2/issues/170
\makeatletter
\setlength{\@fptop}{5pt} % Set distance from top of page to first float
\makeatother
% ---

% ---
% Possibilita criação de Quadros e Lista de quadros.
% Ver https://github.com/abntex/abntex2/issues/176
%
\newcommand{\quadroname}{Quadro}
\newcommand{\listofquadrosname}{Lista de quadros}

\newfloat[chapter]{quadro}{loq}{\quadroname}
\newlistof{listofquadros}{loq}{\listofquadrosname}
\newlistentry{quadro}{loq}{0}

% configurações para atender às regras da ABNT
\setfloatadjustment{quadro}{\centering}
\counterwithout{quadro}{chapter}
\renewcommand{\cftquadroname}{\quadroname\space} 
\renewcommand*{\cftquadroaftersnum}{\hfill--\hfill}

\setfloatlocations{quadro}{hbtp} % Ver https://github.com/abntex/abntex2/issues/176
% ---

% --- 
% Espaçamentos entre linhas e parágrafos 
% --- 

% O tamanho do parágrafo é dado por:
\setlength{\parindent}{1.3cm}

% Controle do espaçamento entre um parágrafo e outro:
\setlength{\parskip}{0.2cm}  % tente também \onelineskip

% ---
% compila o indice
% ---
\makeindex
% ---

%\renewcommand{\ABNTEXchapterfont}{\fontfamily{times}\fontseries{b}\selectfont}
%\renewcommand{\ABNTEXchapterfontsize}{\HUGE}
%\renewcommand{\ABNTEXchapterfont}{\rmfamily\bfseries}
%\setsecheadstyle{\bfseries \Large}
\renewcommand{\ABNTEXchapterfontsize}{\LARGE}

% ---
% Informações de dados para CAPA e FOLHA DE ROSTO
% ---
\titulo{Título do Trabalho Acadêmico}
\autor{Angela Barreto\\ Patricia Bandeira}
\local{Corumbá, MS}
\data{2019}
\orientador{Vagner da Silva Bezerra}
\coorientador{Nome do Coorientador}
\instituicao{
	Instituto Federal de Mato Grosso do Sul \break Curso Técnico Integrado em Informática}
\tipotrabalho{monografia}
% O preambulo deve conter o tipo do trabalho, o objetivo, 
% o nome da instituição e a área de concentração 
\preambulo{Trabalho de Conclusão de Curso apresentado ao curso de nível médio integrado de Informática do Instituto Federal de Educação, Ciência e Tecnologia do Mato Grosso do Sul, em cumprimento às exigências legais como requisito parcial à obtenção do título de Técnico em Informática.}
% ---

% ----
% Início do documento
% ----
\begin{document}
\pagestyle{IFMS}

% Seleciona o idioma do documento (conforme pacotes do babel)
%\selectlanguage{english}
\selectlanguage{brazil}

% Retira espaço extra obsoleto entre as frases.
\frenchspacing 

% ----------------------------------------------------------
% ELEMENTOS PRÉ-TEXTUAIS
% ----------------------------------------------------------
% \pretextual

% ---
% Capa
% ---
\imprimircapa
% ---

% ---
% Folha de rosto
% (o * indica que haverá a ficha bibliográfica)
% ---
\imprimirfolhaderosto
% ---


% ---
% Inserir a ficha bibliografica
% ---

% Isto é um exemplo de Ficha Catalográfica, ou ``Dados internacionais de
% catalogação-na-publicação''. Você pode utilizar este modelo como referência. 
% Porém, provavelmente a biblioteca da sua universidade lhe fornecerá um PDF
% com a ficha catalográfica definitiva após a defesa do trabalho. Quando estiver
% com o documento, salve-o como PDF no diretório do seu projeto e substitua todo
% o conteúdo de implementação deste arquivo pelo comando abaixo:
%
% \begin{fichacatalografica}
%     \includepdf{fig_ficha_catalografica.pdf}
% \end{fichacatalografica}

\begin{fichacatalografica}
	\sffamily
	\vspace*{\fill}					% Posição vertical
	\begin{center}					% Minipage Centralizado
	\fbox{\begin{minipage}[c][8cm]{13.5cm}		% Largura
	\small
	\imprimirautor
	%Sobrenome, Nome do autor
	
	\hspace{0.5cm} \imprimirtitulo  / \imprimirautor. --
	\imprimirlocal, \imprimirdata-
	
	\hspace{0.5cm} \thelastpage p. : il. (algumas color.) ; 30 cm.\\
	
	\hspace{0.5cm} \imprimirorientadorRotulo~\imprimirorientador\\
	
	\hspace{0.5cm}
	\parbox[t]{\textwidth}{\imprimirtipotrabalho~--~\imprimirinstituicao,
	\imprimirdata.}\\
	
	\hspace{0.5cm}
		1. Palavra-chave1.
		2. Palavra-chave2.
		2. Palavra-chave3.
		I. Orientador.
		II. Universidade xxx.
		III. Faculdade de xxx.
		IV. Título 			
	\end{minipage}}
	\end{center}
\end{fichacatalografica}
% ---

% ---

% ---
% Inserir folha de aprovação
% ---

% Isto é um exemplo de Folha de aprovação, elemento obrigatório da NBR
% 14724/2011 (seção 4.2.1.3). Você pode utilizar este modelo até a aprovação
% do trabalho. Após isso, substitua todo o conteúdo deste arquivo por uma
% imagem da página assinada pela banca com o comando abaixo:
%
% \begin{folhadeaprovacao}
% \includepdf{folhadeaprovacao_final.pdf}
% \end{folhadeaprovacao}
%
\begin{folhadeaprovacao}

  \begin{center}
    {\ABNTEXchapterfont\large\imprimirautor}

    \vspace*{\fill}\vspace*{\fill}
    \begin{center}
      \ABNTEXchapterfont\bfseries\Large\imprimirtitulo
    \end{center}
    \vspace*{\fill}
    
    \hspace{.45\textwidth}
    \begin{minipage}{.5\textwidth}
        \imprimirpreambulo
    \end{minipage}%
    \vspace*{\fill}
   \end{center}
        
   Trabalho aprovado. \imprimirlocal, 24 de novembro de 2012:

   \assinatura{\textbf{\imprimirorientador} \\ Orientador} 
   \assinatura{\textbf{Professor} \\ Convidado 1}
   \assinatura{\textbf{Professor} \\ Convidado 2}
   %\assinatura{\textbf{Professor} \\ Convidado 3}
   %\assinatura{\textbf{Professor} \\ Convidado 4}
      
   \begin{center}
    \vspace*{0.5cm}
    {\large\imprimirlocal}
    \par
    {\large\imprimirdata}
    \vspace*{1cm}
  \end{center}
  
\end{folhadeaprovacao}
% ---

% ---
% Dedicatória
% ---
\begin{dedicatoria}
	\vspace*{\fill}
	\centering
	\noindent
	\textit{ Este trabalho é dedicado às crianças adultas que,\\
		quando pequenas, sonharam em se tornar cientistas.} \vspace*{\fill}
\end{dedicatoria}
% ---

% ---
% Agradecimentos
% ---
\begin{agradecimentos}
	
\end{agradecimentos}
% ---

% ---
% Epígrafe
% ---
\begin{epigrafe}
	\vspace*{\fill}
	\begin{flushright}
		\textit{``Não vos amoldeis às estruturas deste mundo, \\
			mas transformai-vos pela renovação da mente, \\
			a fim de distinguir qual é a vontade de Deus: \\
			o que é bom, o que Lhe é agradável, o que é perfeito.\\
			(Bíblia Sagrada, Romanos 12, 2)}
	\end{flushright}
\end{epigrafe}
% ---

% ---
% RESUMOS
% ---

% resumo em português
\setlength{\absparsep}{18pt} % ajusta o espaçamento dos parágrafos do resumo
\begin{resumo}
	
	Tendo em vista que há poucas informações disponíveis sobre as diversas espécies de plantas que predominam na região pantaneira, foi pensado em um meio de dispor tais informações a fim de auxiliar em pesquisas futuras e ofertar um conhecimento a mais para a população residente na região.
	
	A disponibilização de informações ocorrerá por meio de uma aplicação mobile em forma de catálogo, fazendo o uso de uma nova tecnologia denominada React Native, pois ela permite a criação de aplicativos híbridos, ou seja, para android e IOS.
	
	Foi escolhido o desenvolvimento de uma aplicação devido a alta utilização de aparelhos celulares nos dias atuais. A aplicação irá conter informações gerais como o nome científico e popular, ocorrência, distribuição geográfica, grau de ameaça,época de floração.
	
	Espera-se, que com o desenvolvimento da aplicação, o  catálogo mobile possa contribuir com a população como uma forma de adquirir mais conhecimentos sobre a flora pantaneira e também auxiliar em pesquisas no ambiente acadêmico. 25 de maio de 2019


 \textbf{Palavras-chave}: latex. abntex. editoração de texto.
\end{resumo}

% resumo em inglês
\begin{resumo}[Abstract]
 \begin{otherlanguage*}{english}
   This is the english abstract.

   \vspace{\onelineskip}
 
   \noindent 
   \textbf{Keywords}: latex. abntex. text editoration.
 \end{otherlanguage*}
\end{resumo}

% ---
% inserir lista de ilustrações
% ---
\pdfbookmark[0]{\listfigurename}{lof}
\listoffigures*
\cleardoublepage
% ---

% ---
% inserir lista de quadros
% ---
\pdfbookmark[0]{\listofquadrosname}{loq}
\listofquadros*
\cleardoublepage
% ---

% ---
% inserir lista de quadros
% ---
\pdfbookmark[0]{\listofquadrosname}{loq}
\listofquadros*
\cleardoublepage
% ---

% ---
% inserir lista de abreviaturas e siglas
% ---
\begin{siglas}
	\item[ABNT] Associação Brasileira de Normas Técnicas
	\item[abnTeX] ABsurdas Normas para TeX
\end{siglas}
% ---

% ---
% inserir lista de símbolos
% ---
\begin{simbolos}
	\item[$ \Gamma $] Letra grega Gama
	\item[$ \Lambda $] Lambda
	\item[$ \zeta $] Letra grega minúscula zeta
	\item[$ \in $] Pertence
\end{simbolos}
% ---

% ---
% inserir o sumario
% ---
\pdfbookmark[0]{\contentsname}{toc}
\tableofcontents*
\cleardoublepage
% ---

% ----------------------------------------------------------
% ELEMENTOS TEXTUAIS
% ----------------------------------------------------------
\textual

% ----------------------------------------------------------
% Introdução (exemplo de capítulo sem numeração, mas presente no Sumário)
% ----------------------------------------------------------
\chapter{Introdução}
% ----------------------------------------------------------




O Pantanal faz parte  um dos seis biomas continentais brasileiros e segundo Munick Suçuarana, possui uma área com cerca de 250 mil $Km^{2}$, no qual abrange três países: Bolívia, Paraguai e Brasil, em que aproximadamente 62\% se encontra neste último. Localizando-se nos estados de Mato Grosso e Mato Grosso do Sul, tem como parte da sua hidrografia os rios Apa, Cuiabá, Miranda, Paraguai e seus diversos afluentes que formam amplas áreas alagadiças.

A flora pantaneira é formada por uma variedade de plantas dos biomas: Cerrado, Amazônia e Mata Atlântica, e contém poucas espécies nativas do Pantanal.Segundo a EMBRAPA, devido a sua flora diversificada, as plantas se organizam de forma fitogeográfica com as características específicas do bioma de origem.

Pott\&Pott diz que há pouca documentação sobre a fitogeografia da flora pantaneira e surge a necessidade de pesquisas sobre a vegetação. Pensando nisso este trabalho tem por finalidade facilitar a procura de informações a respeito do complexo pantaneiro, realizando a catalogação das plantas e disponibilizando os dados em uma aplicação mobile.

Para o desenvolvimento da aplicação será utilizado a tecnologia React Native.



% ----------------------------------------------------------
% PARTE
% ----------------------------------------------------------
\part{Fundamentação Teórica}
% ----------------------------------------------------------

% ---
% Capitulo com exemplos de comandos inseridos de arquivo externo 
% ---
%\include{configuracoes/Modelo_TCC_IFMS_comandos}

% ---

\chapter{Fundamentação Teórica}\label{cap_trabalho_academico}


\section{Objetivos}
A finalidade é desenvolver uma aplicação mobile, com o intuito de auxiliar pesquisas sobre plantas encontradas na região pantaneira.

\subsection{Objetivos Gerais}
Disponibilizar informações referentes a plantas encontradas no pantanal.

\subsection{Objetivos Específicos}
Definir plantas comuns encontradas no pantanal;

Definir a plataforma para desenvolvimento android;

Definir framework a ser utilizada no backend;

Montar o protótipo;

Validar o protótipo;

Desenvolver o aplicativo móvel para Android.


% ----------------------------------------------------------
% PARTE
% ----------------------------------------------------------
\part{Referenciais teóricos}
% ----------------------------------------------------------

% ---
% Capitulo de revisão de literatura
% ---
\chapter{Referenciais Teóricos}
% ---

% ---
\section{O Bioma Pantanal}

	\subsection{Características}
	
		Este tópico visa conceitualizar as características do Bioma Pantanal,  apresentando a definição de alguns fatores, como o bioma, clima predominante, a sua área ocupacional, entre outros.
		 		 
 		O termo Bioma, segundo Colinvaux (1993), foi proposto por Shelford, que do grego, Bio significa vida e  Oma, significa massa. No ponto de vista de Font Quer (1953), a mesma definição foi criado pelo pesquisador Clements. O diferencial entre um autor e outro, está relacionado com caracterização da palavra formação e da palavra bioma, visto que antes a formação referia-se apenas à vegetação, e bioma ao conjunto de vegetação com associação da fauna. Agora, nesse novo termo de Font Quer, foi totalmente incluso a fauna. Já no glossário do livro de Clements (1949), encontra-se a definição para bioma: “/textit {Biome- A community of plants and animals, usually of the rank a formation: a biotic community}”.
 		 
 		O Pantanal é um dos seis biomas brasileiros, considerado um dos mais deslumbrantes e menos afetado pelos impactos ambientais, em relação aos demais, do qual ocupa cerca de 25\% da região de Mato Grosso do Sul e 7\% em Mato Grosso, totalizando em aproximadamente 50 mil quilômetros quadrados no Brasil (IBGE,2004), possuindo uma ampla área vegetacional, que, por motivos biológicos, encontram-se nele características de outros biomas, como o Cerrado, a Amazônia, a Mata Atlântica e o Chaco boliviano. No verão o clima é quente e com temperatura média em torno de $32\,^{\circ}\mathrm{C}$, com o inverno frio e seco, com média em torno de $21\,^{\circ}\mathrm{C}$ (Portal do Pantanal, 2019). A temperatura, o clima e o solo  acabam influenciando no desenvolvimento de algumas espécies de plantas que predominam o Bioma, havendo distribuição da mesma em um ambiente que mais favorece seu desenvolvimento.
 		 
 		\begin{citacao}
 				O Pantanal mato-grossense é constituído por um complexo ou mosaico de diferentes biomas florestais de hidro biomas e helô-biomas (carandazais, paratudais), savânicos de piro-peinobiomas (cerrados das cordilheiras entre lagoas), florestais de litobiomas (florestas tropicais estacionais caducifólias sobre afloramentos rochosos e solos rasos), campestres de hidro-helobiomas (campos inundáveis), em meio a rios, lagoas de água doce (baias), lagoas de água salobra e alcalina (salinas) etc., todos pertencentes ao Zonobioma II.\cite{coutinho2006conceito} 
 		\end{citacao}
 	
		
		A região pantaneira apresenta uma vegetação de fácil identificação, ou seja, não é homogênea, tendo a flora diferenciada, de acordo com o solo e a altitude. Podemos analisar a vegetação do Cerrado, que apresenta árvores de porte médio, entre arbustos e gramíneas. Os capões de mato são encontrados acima das áreas inundáveis e, consequentemente, a paisagem do Pantanal sofre periodicamente mudanças, em épocas de cheia, o rio transborda, alagando campos e morros isolados, formando pequenas ilhas que servem como abrigo para os animais que foram atingidos. Na seca, os rios retomam seus leitos naturais,  revelando os campos conforme a água vai baixando \cite{eco}.
					
	
	\subsection{Fitogeografia do Pantanal} 
	
		Este tópico visa apresentar o estudo da distribuição geográfica dos fatores históricos e biológicos que determinaram o surgimento do Pantanal, por meio de pesquisas realizadas por \citeonline{ab2006brasil} e \citeonline{pott2009vegetaccao}.
		
		O Pantanal é um bioma que possui uma ampla área vegetacional e também características de outros biomas, que por motivos biológicos encontram-se nele características de outros biomas, como o Cerrado, a Amazônia, Mata Atlântica e o Chaco Boliviano, tornando-se assim fontes de estudos e pesquisas.
		
		Segundo pesquisas realizadas por Célia de Souza e Juberto de Souza, o Pantanal Mato-Grossense é considerado a maior planície alagada contínua do mundo, possuindo em território brasileiro cerca de 140.000 $km^{2}$,  situado nos estados de Mato Grosso e Mato Grosso do Sul.
		
		Várias teorias surgiram no decorrer do tempo para explicar o surgimento do Pantanal. De acordo com \citeonline{ab2006brasil}, geólogo que desenvolveu estudos sobre o espaço brasileiro, pode-se dizer que as informações obtidas referente a formação do Pantanal, demoraram milhares de anos para serem coletadas, passou por processos de pesquisas e análises para obter  uma resposta que esclareça a origem e a evolução do Pantanal.
		
		Uma das teorias, a defendida por Francis Ruellan (1952), referencia-se a Depressão  do Pantanal como um exemplo de \textit {boutonnière}, uma linguagem geomorfológica francesa, que tem como significado “casa de botão”, que busca identificar muitas estruturas das vegetações que foram modificadas e escavadas ao longo do tempo em terreno Pré-Cambriano, que, por consequências dos processos erosivos, causaram esvaziamento do terreno. Segundo a teoria dos refúgios e Redutos, diz que o espaço fisiográfico do Pantanal, iniciou-se devido a uma reativação tectônica, no qual causou o rebaixamento das planícies, causando um preenchimento detrítico de uma bacia de sedimentação, ao longo do tempo, com o desenvolvimento e adaptação da vasta abóbada regional e de terrenos antigos, consequências da erosão, originou-se a depressão pantaneira \cite{ab2006brasil}. Antes do surgimento da denominação utilizada nos dias atuais, o Pantanal era considerado uma vasta abóbada de escudo, que servia como fornecimento detrítico para as bacias do Grupo Bauru e Parecis.
		
		\begin{citacao}
			Na atualidade o pantanal constitui uma extensa planície de acumulação, com topografia plana e alagada periodicamente, tendo o rio Paraguai e seus afluentes o principal meio de transporte de água e sedimentos \cite{souza2006origem}.
		\end{citacao}
		
		Segundo \citeonline{pott2009vegetaccao}, referente a Flora do Pantanal, não havia sido levantado nenhum tipo de espécies para caracterizar a origem da sua formação. Ao realizarem coletas botânicas para serem armazenadas em herbários e para a realização de listagem florística são cerca de 50\% de espécies de ampla distribuição, 20\% do Cerrado e 20\% correspondem a outras origens.
		
		O Chaco boliviano se localiza ao sudeste do Pantanal, nas sub-regiões de Porto Murtinho e Nabileque, no estado de Mato Grosso do Sul. O Cerrado predomina na parte leste e a vegetação com características da Amazônia são encontradas junto aos rios, em partes baixas, situado no oeste.
		
		O Pantanal tem a vegetação como um dos seus principais fatores atrativos, possuindo cerca de 2.000 espécies de fanerógamas, incluindo 200 exóticas, tendo como as principais famílias: leguminosas e gramíneas. As terrestres herbáceas, aproximadamente 1000 espécies \cite{pott2009vegetaccao}.
		
			
		\subsubsection{Primeiros Pesquisadores}
	
			\begin{citacao}
				“As primeiras informações sobre vegetação do Pantanal foram de viagens exploratórias de alguns dos botânicos europeus pioneiros que começaram a desvendar a flora brasileira entre 1825 e 1895, por exemplo, Langsdorff, Riedel, Weddell, Kuntze, Lindman, Malme e Spencer Moore.”\cite{pott2009vegetaccao}
			\end{citacao}
		
		
			O Pantanal possui uma ampla área vegetacional, no qual acabou despertando interesses em alguns pesquisadores europeus, que foram os primeiros a estudarem a vegetação pantaneira. Estudo esse que durou um longo tempo, as informações obtidas referente a formação do Pantanal, demorou milhares de anos para serem coletadas, passou por processos de pesquisas e análises, para obter  uma resposta que esclareça sobre a origem e a evolução do Pantanal. 
			
			Segundo a Teoria dos Refúgios e Redutos de \citeonline{ab2006brasil}, o espaço fisiográfico do Pantanal, iniciou-se devido a uma reativação tectônica, no qual causou o rebaixamento das planícies, causando um preenchimento detrítico de uma bacia de sedimentação,  ao longo do tempo, com o desenvolvimento e adaptação da vasta abóbada regional e de terrenos antigos, consequências da erosão, originou-se a depressão pantaneira.
					
			De acordo com Salis, Suzana Maria et al. (2006) a Flora Pantaneira teve suas primeiras informações registrada no final do século XIX por naturalistas europeus, como o Moore (1895) e Malme (1905), que trabalharam com espécies de algumas famílias (Leguminosae e Vochysiaceae), Veloso (1947) que classificou alguns tipos de vegetações relacionando-as com regiões inundadas e a sua composição de espécies, o levantamento e listagens de algumas espécies ocorrentes no Pantanal foram realizadas por Pott et al. (1986, 1997) e Guarim Neto, a composição florística dos capões localizados no Região do Abobral, no município de Corumbá, estado de Mato Grosso do Sul, foram estudadas por Damasceno Júnior et al (1999). Prance \& Schaller (1982), Ratter et al. (1988), Nascimento \& Cunha (1989) e Cunha (1990) realizaram os trabalhos pioneiros referente aos aspectos fitossociológicos nas regiões de florestas e cerradões do Pantanal. Os estudos das estruturas em cerradões e florestas no Pantanal da Nhecolândia no estado de Mato Grosso do Sul foram realizados por Soares (1997), Salis et al. (1999) e Salis (2000).
		
			Estudos realizados por  Ratter et al. (1996, 2003) e Castro et al. (1999), sobre o Cerrado brasileiro, teve o intuito de levantar dados sobre os aspectos e distribuições das vegetações encontradas no Pantanal. Diante de observações notou-se um padrão geográfico na distribuição da flora, no qual seis grupos florísticos foram reconhecidos após a comparação e análise de 376 áreas do Cerrado, o grupo do Sudeste, (cerrados do Estado de São Paulo, Paraná e Sul de Minas Gerai); Centro-Sudoeste (cerrados do Distrito Federal e de Minas Gerais); Norte-Nordeste (Bahia, Ceará, Norte de Minas Gerais, Maranhão, Piauí e parte do Leste de Tocantins); Centro-Oeste (Mato Grosso, Mato Grosso do Sul, Goiás , parte do Oeste de Tocantins e Sul do Paraná ); mesotrófico-oeste(cerrados de vários estados, principalmente de Rondônia e Mato Grosso do Sul) e um grupo de áreas disjuntas, formado pelas savanas amazônicas \cite{souza2006origem}.
		
	
\section{Vegetações}

	Neste tópico será abordado conteúdos sobre as oito famílias escolhidas para o desenvolvimento deste trabalho, retratando o nome científico, nome popular, época de floração, grau de ameaça, entre outros.

	\subsection{Famílias}
	
		\begin{description}
			\item[Anacardiceae]
			\item[Arecaceae] 
			\item[Bignoniaceae]
			\item[Boraginaceae]
			\item[Ebenaceae]
			\item[Fabaceae]
			\item[Lamiaceae]
			\item[Rubiaceae]
			
		\end{description}
	
	
		\subsubsection{Espécies}
		
\section{Sistemas Operacionais}
		
	De acordo com \citeonline{toledo2016desenvolvimento}, "São chamados smartphones os telefones celulares que oferecem alta capacidade de processamento", ou seja, oferece recursos avançados e segundo \citeonline []{torres2018biblia}, é um computador que pode ser transportado a palma da mão, e como todo computador os smartphones necessitam de sistemas operacionais para funcionar.
		
	O sistema operacional é responsável por gerenciar processos e realizar a comunicação entre aplicação e hardware, além de disponibilizar ao usuário uma interface no computador, de forma que a utilização fique mais agradável. \cite {velloso2014informatica}
			
	Hoje em dia o mercado de dispositivos móveis oferece numerosos modelos de aparelhos, e cada fabricante opta por um determinado sistema operacional. Para este trabalho foram selecionados dois sistemas operacionais o Android da empresa Google e o iOS da Apple. 
		
	\subsection{Android}
			
		Liderado pela Google, o Android é uma Linguagem Open Source, ou seja, de código livre para dispositivos móveis, como é um projeto Open Source a documentação é pública e livre para todos, o objetivo é evitar qualquer falha no sistema através de feedbacks transmitidos por seus usuários. Dessa maneira o Android torna-se um sistema operacional completo com código-fonte personalizável que pode ser enviado para qualquer dispositivo.
				
		%https://source.android.com/
				
	\subsection{IOS}
			
				
			
\section{Desenvolvimento Mobile}

	Este capítulo tem por objetivo apresentar as categorias de desenvolvimento mobile segundo os pesquisadores \citeonline {da2014paradigmas}, \citeonline{emdesafios} e \citeonline{toledo2016desenvolvimento}.

	\begin{citacao}
		“A evolução da tecnologia dos aparelhos celulares permitiu oferecer ao usuário recursos que vão muito além da realização de uma chamada ou do envio de uma mensagem. As melhorias de hardware dos aparelhos celulares permitiram o desenvolvimento de sistemas operacionais mais avançados.”\cite[]{da2014paradigmas}
	\end{citacao}
	
	Com o avanço dos Sistemas operacionais tornou-se factível produzir aplicativos melhores, tornando os aparelhos celulares mais comuns no cotidiano das pessoas devido a comodidade em facilitar a realização de tarefas diárias.
	
	Segundo \citeonline []{emdesafios}, os aplicativos móveis possibilitaram a criação de novos modelos de negócio, em que o usuário pode ter acesso a novos recursos e serviços, por meio de lojas virtuais produzidas por fabricantes de dispositivos móveis. O principal benefício de surgimento de novos modelos é a ligação de serviços e aplicações avançadas de Internet ao aparelho mobile. Entretanto, existem diferentes plataformas tecnológicas para desenvolvimento mobile, incluindo sistemas operacionais e IDE's  (Integrated Development Environment), gerando uma ampla diversidade de soluções disponíveis no mercado. Os aplicativos móveis dependem de uma organização de código específico para que possam ser continuamente executados em diversas plataformas e versões.
	
	\begin{citacao}
		"Limitações de plataforma para distribuição do aplicativo, como, tempo, custo para o desenvolvimento e complexidade das tecnologias necessárias para a sua criação e manutenção são pontos problemáticos em um projeto voltado ao desenvolvimento deste tipo de aplicativo."\cite{emdesafios}
	\end{citacao}
	
	
	Seguindo essa linha de pensamento existem duas maneiras de categorizar
	aplicativos, nativas e web, ambos com seus prós e contras. Há também aplicações multiplataformas que englobam algumas funcionalidades de ambas categorias. Pensando nisso é fundamental saber qual utilizar na hora de planejar um desenvolvimento mobile.
	
	\subsection{Desenvolvimento de Aplicações Nativas}
	
		Aplicativos Nativos são aplicações desenvolvidas para determinados SOs, com Linguagens de programação e IDEs específicas. Cada qual com sua responsabilidade, O sistema operacional é responsável por gerenciar os recursos do dispositivo, a linguagem é utilizada na programação do software e o IDE dispõe ferramentas que ajudam na programação da aplicação.
		
		Com a finalidade de melhorar a interação com o usuário, aplicações nativas utilizam recursos presentes no aparelho,como dispositivos de hardware ()GPS, microfone, magnetômetro e câmera) e software (e-mails,  calendário, contatos, fotografias e telefone), para isto os aplicativos são instalados diretamente no sistema operacional, facilitando o uso da aplicação no modo off-line, caso não haja acesso a internet.\cite{toledo2016desenvolvimento}
		
		Apesar das múltiplas funcionalidades, aplicativos nativos não são fáceis de desenvolver, pois para ser executado em diversos dispositivos é preciso utilizar uma variedade de plataformas e programar de forma que funcione diversas versões. Devido a grande quantidade de plataformas, mantê-lo atualizado é desvantajoso, pois faz-se necessário uma variedade de teste e distribuição para cada versão.
	
	\subsection{Desenvolvimento de Aplicações Web}
	
		Uma aplicação web móvel nada mais é que um site em formato para celulares, que pode ser acessado de um navegador Web instalado no dispositivo. Tal qual um ambiente web tradicional, é desenvolvido utilizando as linguagens HTML para texto estáticos e imagens, CSS para o design da tela e JavaScript que é responsável pelos efeitos na aplicação. Por serem baseados em navegadores web, as aplicações podem ser executadas em quaisquer dispositivos que possuem acesso a internet.\cite{emdesafios}
		
		Segundo \citeonline {emdesafios}, as aplicações web são mais baratas, de fácil manutenção e não ocupam memória no dispositivo,pois, não precisam ser instaladas, o que as torna vantajosas devido ao fato de serem compatíveis com diversas plataformas.
		
		Mesmo com as diversas vantagens, é necessário o acesso constante à internet em uma velocidade satisfatória para que haja uma rápida troca de dados com os servidores, além de que se o sistema precisar utilizar recursos e funcionalidades do dispositivo as aplicações web podem ser consideradas descartadas.\cite{toledo2016desenvolvimento}.
		
		Como as aplicações nativas, aplicativos web necessitam de vários testes para adequar ao tamanho de tela dos dispositivos e as exigências de cada navegador, uma vez que o mercado de smartphones é amplo e cada fabricante faz uso de navegadores que se adaptam melhor ao seus produtos.
			
	\subsection{Desenvolvimento de Aplicações Híbridas}
	
		As  aplicações híbridas é a junção de aplicativos nativos e WEB, pois como os nativos, aplicações híbridas são baixadas diretamente no dispositivo e fazer uso de funcionalidades e recursos do mesmo, por exemplo a câmera, bluetooth e GPS. Assim com aplicativos WEB, os híbridos fazem uso de linguagens HTML, JavaScript e CSS. \cite{tavares2016introduccao}
		
		A vantagem deste tipo de aplicação é a facilidade no desenvolvimento, pois é preciso apenas um código para ser compilado para diversas plataformas, poupando tempo para o  desenvolvedor, dado que não se faz necessário a programação em várias linguagens.
	
\section{Frameworks}
	
	De acordo com \citeonline {sawaya2002dicionario}, autora do dicionário de Informática \& Internet, framework é um “conjunto de elementos e suas interligações constituindo a base de um sistema ou projeto”.
	
	Frameworks servem como ponto de partida no desenvolvimento de um projeto, pois reduz o código-fonte reaproveitando métodos, classes e funções presentes no projeto. Além disso, os frameworks fornecem um design padrão e ajustável, para tornar a interface agradável.\cite{gabardo2017laravel}
	
	Segundo \citeonline {gabardo2017laravel}, existem diversos frameworks para quase todas as linguagens, de forma que o desenvolvedor possa escolher a que melhor atende as necessidades do sistema . Para este trabalho foram escolhidas duas Frameworks, para desenvolvimento mobile e web.
	

	\subsection{Laravel}
	
		O framework Laravel foi lançado em 2011, foi desenvolvido com o intuito de agilizar a criação de aplicações web. É um framework, que atualmente vem sendo cada vez mais utilizado pelos os programadores, pois um dos principais fatores que o tornam “especial”, é por possuir uma fácil instalação e configuração. 
		
		\begin{citacao}
			Verificou-se os tipos especiais de teste que o framework Laravel fornece, como a parametrização de dados que, além de facilitar o desenvolvimento de software em relação aos relacionamentos do banco de dados, facilita o processo de automação de teste de software.\cite{pelizza2018estudo}
		\end{citacao}
		
		
		O Laravel é baseado na linguagem PHP, e uma de suas desvantagens é a necessidade de requisitos/extensões para um bom funcionamento. Porém, o Laravel possui uma arquitetura MVC (\textit{Model-View-Controller}) e tem como aspecto principal, o de auxiliar no desenvolvimento de aplicações seguras e de alto desempenho de forma ágil e simplificada, com código limpo.\cite{pelizza2018estudo}
	
	\subsection{React Native}
	
		Desenvolvido pelo Facebook, o React foi criado para produzir interfaces de usuário, utilizando a linguagem JavaScript. O React divide-se em duas categorias, ambas Open Source, ReactJS e React Native. Lançada em 2013, o ReactJS é utilizado para desenvolvimento web e oferece uma rápida resposta às entradas do usuário, devido a utilização do JavaScript \cite{vilete2018frameworks}.
		
		De acordo com \citeonline{vilete2018frameworks}, o React Native possui estrutura de desenvolvimento híbrido para aplicações móveis. Lançado em 2015, o framework utiliza recursos nativos de ambos, pois ao ser executado compila o código-fonte paras as linguagens Java e Swift, linguagens de programação das plataformas Android e iOS respectivamente.
	
\section{Ambientes de Desenvolvimento}
	
	\subsection{Visual Studio Code}
	
	\subsection{MySql Workbench}
	
\section{Linguagens de Programção}	
	
	
	\subsection{PHP}
		
		\subsubsection{Composer}
				
	\subsection{JavaScript}
		
	\subsection{HTML5}
		
	\subsection{CSS3}
		
		%\subsection{node e npm}
	
 











% ----------------------------------------------------------
% PARTE
% ----------------------------------------------------------
\part{Resultados}
% ----------------------------------------------------------

% ---
% primeiro capitulo de Resultados
% ---
\chapter{Lectus lobortis condimentum}
% ---

% ---
\section{Vestibulum ante ipsum primis in faucibus orci luctus et ultrices
posuere cubilia Curae}
% ---



% ---
% segundo capitulo de Resultados
% ---
\chapter{Nam sed tellus sit amet lectus urna ullamcorper tristique interdum
elementum}
% ---

% ---
\section{Pellentesque sit amet pede ac sem eleifend consectetuer}
% ---



% ----------------------------------------------------------
% Finaliza a parte no bookmark do PDF
% para que se inicie o bookmark na raiz
% e adiciona espaço de parte no Sumário
% ----------------------------------------------------------
\phantompart

% ---
% Conclusão
% ---
\chapter{Conclusão}
% ---



% ----------------------------------------------------------
% ELEMENTOS PÓS-TEXTUAIS
% ----------------------------------------------------------
\postextual
% ----------------------------------------------------------

% ----------------------------------------------------------
% Referências bibliográficas
% ----------------------------------------------------------
\bibliography{elementos/referencias_bibliograficas}

% ----------------------------------------------------------
% Glossário
% ----------------------------------------------------------
%
% Consulte o manual da classe abntex2 para orientações sobre o glossário.
%
%\glossary

% ----------------------------------------------------------
% Apêndices
% ----------------------------------------------------------

% ---
% Inicia os apêndices
% ---
\begin{apendicesenv}

% Imprime uma página indicando o início dos apêndices
\partapendices

% ----------------------------------------------------------
\chapter{Quisque libero justo}
% ----------------------------------------------------------

\lipsum[50]

% ----------------------------------------------------------
\chapter{Nullam elementum urna vel imperdiet sodales elit ipsum pharetra ligula
ac pretium ante justo a nulla curabitur tristique arcu eu metus}
% ----------------------------------------------------------
\lipsum[55-57]

\end{apendicesenv}
% ---


% ----------------------------------------------------------
% Anexos
% ----------------------------------------------------------

% ---
% Inicia os anexos
% ---
\begin{anexosenv}

% Imprime uma página indicando o início dos anexos
\partanexos

% ---
\chapter{Tabela de Plantas Escolhidas}
% ---

	\begin{table}[h]
		
		\caption{Relação de Plantas}
		\begin{tabular}{ccccc}
			\hline
			Nome		  	 & Nome Científico 				   			& Família 		& Época de 			 & Grau de			\\
			Popular			 & 											&				& Floração			& Ameaça			\\
			% Note a separação de col. e a quebra de linhas
			\hline                               % para uma linha horizontal
			Acuri   		 & \textit{Attalea phalerata}	   			& Arecaceae 	& ago-jan 			& LC				\\
							 & Mart. ex Spreng				   			&				&				    &					\\
							
			Amburana		 & \textit{Amburana cearensis} 	   			& Fabaceae 		& mar-jun 			& NT				\\
							 & (Allemão) A.C.Sm				   			&				&				    &					\\
			
			Angico 			 & \textit{Albizia niopoides} 	   			& Fabaceae 		& out-jan 			& LC 				\\
							 & (Spruce ex Benth.) Burkart 	   			&				&					&					\\
								
			Aroeira 		 & \textit{Myracrodruon urundeuva} 			& Anacardiaceae	& ago-set 			& LC 				\\
							 &  Allemão									&				&				    &					\\
			
			Bocaiúva 		 & \textit{Acrocomia aculeata}      		& Arecaceae 	& out-fev 			& NE 				\\
							 &  (Jacq) Lodd. ex Mart.		    		&				&				    &					\\
			
			Cajá 			 & \textit{Spondias mombin}		    		& Anacardiaceae & out-nov 			& NE 				\\
							 & L. 						  	    		&				&				    &					\\
			
			Caju 			 & \textit{Anacardium humile}  	    		& Anacardiaceae & ago-nov 			& LC 				\\
							 &  A.St.-Hil.					    		&				&			    	&					\\
			
			Caqui do Cerrado & \textit{Diospyros hispida}  	    		& Ebenaceae 	& ago-nov 			& LC 				\\
							 &  A.DC						    		&				&				   	&					\\
			
			Caranda 		 & \textit{Copernicia alba}  	    		& Arecaceae 	& jul-dez 			& NE 				\\
							 & Morong ex Morong \& Britton	    		&				&					&					\\
			
			Gonçalo 		 & \textit{Astronium fraxinifolium}			& Anacardiaceae & jul-set 			& LC 				\\
							 &  Schott.									&				&				    &					\\
			
			Ipê-Amarelo		 & \textit{Tabebuia aurea}   				& Bignoniaceae 	& ago-out 			& NE 				\\
							 & (Silva Manso) Benth. \&	Hook.f.			&				&				    &					\\
							 & ex S.Moore								&				&				    &					\\
						
			Ipê-Branco 		 & \textit{Tabebuia roseoalba}  			& Bignoniaceae 	& ago-dez			& NE 				\\
							 & (Ridl.) Sandwith 						&				&				    &					\\
			
			Ipê-Verde 		 & \textit{Cybistax antisyphilitica}		& Bignoniaceae  & ago-mar 			& NE 				\\
							 & (Mart) Mart 								&				&				    &					\\
			
			Jenipapo 		 & \textit{Genipa americana}  				& Rubiaceae 	& set-dez 			& LC 				\\
							 & L. 										&				&				    &					\\
			
			Louro Preto 	 & \textit{Cordia glabrata}  				& Boraginaceae 	& ago-out 			& NE 				\\
							 & (Mart.) A.DC. 							&				&				    &					\\
			
			Passarinho 		 & \textit{Pterogyne nitens}  				& Fabaceae 		& fev-ago 			& LC 				\\
							 & Tul										&				&				    &					\\
			
			Pata de vaca 	 & \textit{Bauhinia rufa}	  				& Fabaceae 		& mai-nov 			& NE 				\\
							 &  (Bong.) Steud.							&				&				    &					\\
			
			Sucupira 		 & \textit{Bowdichia virgilioides}		  	& Fabaceae 		& jun-ago 			& NT 				\\
							 & Kunth									&				&				    &					\\
			
			Tarumã           & \textit{Vitex cymosa}	 				& Lamiaceae 	& set-nov 			& NE 				\\
							 &  Bertero ex Spreng.						& 				&					&					\\
			
			Ximbuva 		 & \textit{Enterolobium contortisiliquum}	& Fabaceae 		& set-nov 			& NE 				\\
							 &  (Vell.) Morong 							&				&					&					\\
			\hline
		\end{tabular}
	\end{table}

% ---
\chapter{Anexo B}
% ---



% ---
\chapter{Anexo C}
% ---



\end{anexosenv}

%---------------------------------------------------------------------
% INDICE REMISSIVO
%---------------------------------------------------------------------
\phantompart
\printindex
%---------------------------------------------------------------------

\end{document}
