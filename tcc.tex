%% abtex2-modelo-trabalho-academico.tex, v-1.9.7 laurocesar
%% Copyright 2012-2018 by abnTeX2 group at http://www.abntex.net.br/ 
%%
%% This work may be distributed and/or modified under the
%% conditions of the LaTeX Project Public License, either version 1.3
%% of this license or (at your option) any later version.
%% The latest version of this license is in
%%   http://www.latex-project.org/lppl.txt
%% and version 1.3 or later is part of all distributions of LaTeX
%% version 2005/12/01 or later.
%%
%% This work has the LPPL maintenance status `maintained'.
%% 
%% The Current Maintainer of this work is the abnTeX2 team, led
%% by Lauro César Araujo. Further information are available on 
%% http://www.abntex.net.br/
%%
%% This work consists of the files abntex2-modelo-trabalho-academico.tex,
%% abntex2-modelo-include-comandos and abntex2-modelo-references.bib
%%

% ------------------------------------------------------------------------
% ------------------------------------------------------------------------
% abnTeX2: Modelo de Trabalho Academico (tese de doutorado, dissertacao de
% mestrado e trabalhos monograficos em geral) em conformidade com 
% ABNT NBR 14724:2011: Informacao e documentacao - Trabalhos academicos -
% Apresentacao
% ------------------------------------------------------------------------
% ------------------------------------------------------------------------

\documentclass[
	% -- opções da classe memoir --
	12pt,				% tamanho da fonte
%	openright,			% capítulos começam em pág ímpar (insere página vazia caso preciso)
	oneside,			% para impressão em recto e verso. Oposto a oneside
	a4paper,			% tamanho do papel. 
	% -- opções da classe abntex2 --
	%chapter=TITLE,		% títulos de capítulos convertidos em letras maiúsculas
	%section=TITLE,		% títulos de seções convertidos em letras maiúsculas
	%subsection=TITLE,	% títulos de subseções convertidos em letras maiúsculas
	%subsubsection=TITLE,% títulos de subsubseções convertidos em letras maiúsculas
	% -- o\part{title}pções do pacote babel --
	english,			% idioma adicional para hifenização
	french,				% idioma adicional para hifenização
	spanish,			% idioma adicional para hifenização
	brazil				% o último idioma é o principal do documento
	]{configuracoes/tcc}

% ---
% Pacotes básicos 
% ---
\usepackage{lmodern}			% Usa a fonte Latin Modern
%\usepackage{times}				% Usa a fonte Times
\usepackage{mathptmx}				% Usa a fonte Times
\usepackage[T1]{fontenc}		% Selecao de codigos de fonte.
\usepackage[utf8]{inputenc}		% Codificacao do documento (conversão automática dos acentos)
\usepackage{indentfirst}		% Indenta o primeiro parágrafo de cada seção.
\usepackage{color}				% Controle das cores
\usepackage{graphicx}			% Inclusão de gráficos
\usepackage{microtype} 			% para melhorias de justificação
\usepackage{varwidth}
\usepackage{array}
% ---
		
% ---
% Pacotes adicionais, usados apenas no âmbito do Modelo Canônico do abnteX2
% ---
\usepackage{lipsum}				% para geração de dummy text
% ---

% ---
% Pacotes de citações
% ---
\usepackage[brazilian,hyperpageref]{backref}	 % Paginas com as citações na bibl
\usepackage[alf]{abntex2cite}	% Citações padrão ABNT
%Pacote para justificar textos
\usepackage{ragged2e}

% --- 
% CONFIGURAÇÕES DE PACOTES
% --- 

% ---
% Configurações do pacote backref
% Usado sem a opção hyperpageref de backref
\renewcommand{\backrefpagesname}{Citado na(s) página(s):~}
% Texto padrão antes do número das páginas
\renewcommand{\backref}{}
% Define os textos da citação
\renewcommand*{\backrefalt}[4]{
	\ifcase #1 %
		Nenhuma citação no texto.%
	\or
		Citado na página #2.%
	\else
		Citado #1 vezes nas páginas #2.%
	\fi}%
% ---

% ---
% Configurações de aparência do PDF final

% alterando o aspecto da cor azul
\definecolor{blue}{RGB}{41,5,195}

% informações do PDF
\makeatletter
\hypersetup{
	%pagebackref=true,
	pdftitle={\@title}, 
	pdfauthor={\@author},
	pdfsubject={\imprimirpreambulo},
	pdfcreator={LaTeX with abnTeX2},
	pdfkeywords={abnt}{latex}{abntex}{abntex2}{trabalho acadêmico}, 
	colorlinks=true,       		% false: boxed links; true: colored links
	linkcolor=black,          	% color of internal links
	citecolor=black,        		% color of links to bibliography
	filecolor=magenta,      		% color of file links
	urlcolor=blue,
	bookmarksdepth=4
}
\makeatother
% --- 

% ---
% Posiciona figuras e tabelas no topo da página quando adicionadas sozinhas
% em um página em branco. Ver https://github.com/abntex/abntex2/issues/170
\makeatletter
\setlength{\@fptop}{5pt} % Set distance from top of page to first float
\makeatother
% ---

% ---
% Possibilita criação de Quadros e Lista de quadros.
% Ver https://github.com/abntex/abntex2/issues/176
%
\newcommand{\quadroname}{Quadro}
\newcommand{\listofquadrosname}{Lista de quadros}

\newfloat[chapter]{quadro}{loq}{\quadroname}
\newlistof{listofquadros}{loq}{\listofquadrosname}
\newlistentry{quadro}{loq}{0}

% configurações para atender às regras da ABNT
\setfloatadjustment{quadro}{\centering}
\counterwithout{quadro}{chapter}
\renewcommand{\cftquadroname}{\quadroname\space} 
\renewcommand*{\cftquadroaftersnum}{\hfill--\hfill}

\setfloatlocations{quadro}{hbtp} % Ver https://github.com/abntex/abntex2/issues/176
% ---

% --- 
% Espaçamentos entre linhas e parágrafos 
% --- 

% O tamanho do parágrafo é dado por:
\setlength{\parindent}{1.3cm}

% Controle do espaçamento entre um parágrafo e outro:
\setlength{\parskip}{0.2cm}  % tente também \onelineskip

% ---
% compila o indice
% ---
\makeindex
% ---

%\renewcommand{\ABNTEXchapterfont}{\fontfamily{times}\fontseries{b}\selectfont}
%\renewcommand{\ABNTEXchapterfontsize}{\HUGE}
%\renewcommand{\ABNTEXchapterfont}{\rmfamily\bfseries}
%\setsecheadstyle{\bfseries \Large}
\renewcommand{\ABNTEXchapterfontsize}{\LARGE}

% ---
% Informações de dados para CAPA e FOLHA DE ROSTO
% ---
\titulo{Mundo das Plantas: Catálogo Mobile de Plantas comuns no Bioma Pantanal}
\autor{Angela Barreto\\ Patricia Bandeira}
\local{Corumbá, MS}
\data{2019}
\orientador{Vagner da Silva Bezerra}
\coorientador{Izabelli dos Santos Ribeiro}
\instituicao{
	Instituto Federal de Mato Grosso do Sul \break Curso Técnico Integrado em Informática}
\tipotrabalho{monografia}
% O preambulo deve conter o tipo do trabalho, o objetivo, 
% o nome da instituição e a área de concentração 
\preambulo{Trabalho de Conclusão de Curso apresentado ao curso de nível médio integrado de Informática do Instituto Federal de Educação, Ciência e Tecnologia do Mato Grosso do Sul, em cumprimento às exigências legais como requisito parcial à obtenção do título de Técnico em Informática.}
% ---

% ----
% Início do documento
% ----
\begin{document}
\pagestyle{IFMS}

% Seleciona o idioma do documento (conforme pacotes do babel)
%\selectlanguage{english}
\selectlanguage{brazil}

% Retira espaço extra obsoleto entre as frases.
\frenchspacing 

% ----------------------------------------------------------
% ELEMENTOS PRÉ-TEXTUAIS
% ----------------------------------------------------------
% \pretextual

% ---
% Capa
% ---
\imprimircapa
% ---

% ---
% Folha de rosto
% (o * indica que haverá a ficha bibliográfica)
% ---
\imprimirfolhaderosto
% ---


\include{elementos/pre_textuais/ficha_catalografica}

\include{elementos/pre_textuais/folha_aprovacao}

\include{elementos/pre_textuais/dedicatoria}

\include{elementos/pre_textuais/agradecimentos}

\include{elementos/pre_textuais/epigrafe}

% ---
% RESUMOS
% ---

% resumo em português
\setlength{\absparsep}{18pt} % ajusta o espaçamento dos parágrafos do resumo
\begin{resumo}
	
	Tendo em vista que há poucas informações disponíveis sobre as diversas espécies de plantas que predominam na região pantaneira, foi pensado em um meio de dispor tais informações a fim de auxiliar em pesquisas futuras e ofertar um conhecimento a mais para a população residente na região.
	
	A disponibilização de informações ocorrerá por meio de uma aplicação mobile em forma de catálogo, fazendo o uso de uma nova tecnologia denominada React Native, pois ela permite a criação de aplicativos híbridos, ou seja, para android e IOS.
	
	Foi escolhido o desenvolvimento de uma aplicação devido a alta utilização de aparelhos celulares nos dias atuais. A aplicação irá conter informações gerais como o nome científico e popular, ocorrência, distribuição geográfica, grau de ameaça,época de floração.
	
	Espera-se, que com o desenvolvimento da aplicação, o  catálogo mobile possa contribuir com a população como uma forma de adquirir mais conhecimentos sobre a flora pantaneira e também auxiliar em pesquisas no ambiente acadêmico. 25 de maio de 2019


 \textbf{Palavras-chave}: latex. abntex. editoração de texto.
\end{resumo}

% resumo em inglês
\begin{resumo}[Abstract]
 \begin{otherlanguage*}{english}
   This is the english abstract.

   \vspace{\onelineskip}
 
   \noindent 
   \textbf{Keywords}: latex. abntex. text editoration.
 \end{otherlanguage*}
\end{resumo}

\include{elementos/pre_textuais/lista_ilustracoes}

\include{elementos/pre_textuais/lista_quadros}

\include{elementos/pre_textuais/lista_tabelas}

\include{elementos/pre_textuais/lista_abreviaturas_siglas}

\include{elementos/pre_textuais/lista_simbolos}

\include{elementos/pre_textuais/sumario}

% ----------------------------------------------------------
% ELEMENTOS TEXTUAIS
% ----------------------------------------------------------
\textual

% ----------------------------------------------------------
% Introdução (exemplo de capítulo sem numeração, mas presente no Sumário)
% ----------------------------------------------------------
\chapter{Introdução}
% ----------------------------------------------------------


O Pantanal é um dos seis biomas brasileiros, ainda que seja o menor é considerado  pouco afetado pelos impactos ambientais em comparação aos demais. Por possuir uma ampla flora, são encontradas inúmeros tipos de espécies diferentes, podendo estas pertencer a mais de um bioma, como o Cerrado, Amazônia, Mata Atlântica e o Chaco Boliviano.

O surgimento do Bioma Pantanal segundo o geólogo \citeonline{ab2006brasil}, ocorreu por meio de abalos tectônicos, o qual causou uma enorme erosão causando o rebaixamento das planícies e levando ao preenchimento detrítico das bacias sedimentares, que ficou denominada como Abóbada. Ao longo do tempo conforme a adaptação e desenvolvimento da vasta Abóbada originou-se o Bioma Pantanal, denominado na época Depressão Pantaneira. 

O Bioma Pantanal é um importante fator a ser estudado, podendo ser fonte de pesquisas acadêmicas e científicas. Porém, são poucos os meios em que são disponibilizados  conteúdos relacionados a flora e fauna do Bioma Pantanal. O presente trabalho tem como objetivo abranger temáticas referente a flora Pantaneira, dos quais foram selecionadas 20 plantas encontradas nesta região, as mesmas foram retiradas do site Web Ambiente e Flora do Brasil 2020, ambas possuem um dos maiores bancos de dados que fornecem informações sobre a flora de todos os biomas brasileiros. 

Tendo definido o objetivo do trabalho,o meio de disponibilizar conteúdos das espécies elegidas, será mediante ao desenvolvimento de uma aplicação mobile híbrida, que seguirá o modelo de um catálogo, o qual disponibilizará informações relevantes sobre a planta, como nome científico, nome popular, área ocupacional, grau de ameaça  e distribuição.

A aplicação tem seu desenvolvimento baseado em uma nova tecnologia, denominada React Native, um framework mobile baseado no React desenvolvido pela empresa Facebook. Este permite o desenvolvimento mobile híbrido, de forma que possa ser utilizado em dispositivos com Sistema Operacional Android e iOS.


%O Pantanal faz parte  um dos seis biomas continentais brasileiros e segundo Munick Suçuarana, possui uma área com cerca de 250 mil $Km^{2}$, no qual abrange três países: Bolívia, Paraguai e Brasil, em que aproximadamente 62\% se encontra neste último. Localizando-se nos estados de Mato Grosso e Mato Grosso do Sul, tem como parte da sua hidrografia os rios Apa, Cuiabá, Miranda, Paraguai e seus diversos afluentes que formam amplas áreas alagadiças.

%A flora pantaneira é formada por uma variedade de plantas dos biomas: Cerrado, Amazônia e Mata Atlântica, e contém poucas espécies nativas do Pantanal.Segundo a EMBRAPA, devido a sua flora diversificada, as plantas se organizam de forma fitogeográfica com as características específicas do bioma de origem.

%Pott\&Pott diz que há pouca documentação sobre a fitogeografia da flora pantaneira e surge a necessidade de pesquisas sobre a vegetação. Pensando nisso este trabalho tem por finalidade facilitar a procura de informações a respeito do complexo pantaneiro, realizando a catalogação das plantas e disponibilizando os dados em uma aplicação mobile.

%Para o desenvolvimento da aplicação será utilizado a tecnologia React Native.



% ----------------------------------------------------------
% PARTE
% ----------------------------------------------------------
\part{Fundamentação Teórica}
% ----------------------------------------------------------

% ---
% Capitulo com exemplos de comandos inseridos de arquivo externo 
% ---
%\include{configuracoes/Modelo_TCC_IFMS_comandos}

% ---

\chapter{Fundamentação Teórica}\label{cap_trabalho_academico}


\section{Objetivos}
A finalidade é desenvolver uma aplicação mobile, com o intuito de auxiliar pesquisas sobre plantas encontradas na região pantaneira.

\subsection{Objetivos Gerais}
Disponibilizar informações referentes a plantas encontradas no pantanal.

\subsection{Objetivos Específicos}
Definir plantas comuns encontradas no pantanal;

Definir a plataforma para desenvolvimento android;

Definir framework a ser utilizada no backend;

Montar o protótipo;

Validar o protótipo;

Desenvolver o aplicativo móvel para Android.


% ----------------------------------------------------------
% PARTE
% ----------------------------------------------------------
\part{Referenciais teóricos}
% ----------------------------------------------------------

% ---
% Capitulo de revisão de literatura
% ---
\chapter{Referenciais Teóricos}
% ---

% ---
\section{O Bioma Pantanal}

	Este capítulo visa apresentar princípios mais relevantes  com a finalidade de explicar o que é o Bioma Pantanal do Dicionário Ambiental.((0))eco,conceituando o significado de Bioma no ponto de vista de alguns pesquisadores citados no artigo de \citeonline []{coutinho2006conceito}.
	
	O Pantanal é um bioma que possui uma ampla área vegetacional, que por motivos biológicos encontram-se nele características de outros biomas, como o Cerrado, a Amazônia, Mata Atlântica e o Chaco Boliviano, tornando-se fontes de estudos e pesquisas.
	
	Situado em grande parte do centro-oeste brasileiro, estendem-se pelo Paraguai, Argentina e Bolívia. Segundo o IBGE (Instituto Brasileiro de Geografia e Estatística), no Centro Oeste do Brasil é ocupado  cerca  de 189 mil quilômetros quadrados, dos quais 63\% na região de Mato Grosso do Sul e 37\% em Mato Grosso.
	
	A análise do clima é quente no verão e com temperatura média em torno de $32\,^{\circ}\mathrm{C}$, sendo frio e seco no inverno com média em torno de $21\,^{\circ}\mathrm{C}$, ocorrendo raramente geada nos meses de julho e agosto (Portal do Pantanal, 2019).	

	\subsection{Características}
	
		
	
	\subsection{Fitogeografia do Pantanal} 
	
		Este capítulo visa apresentar o estudo da distribuição geográfica por meio de pesquisas realizadas por \citeonline {ab2006brasil}, \citeonline {pott2009vegetaccao} e dos fatores históricos e biológicos que a determinaram o surgimento do Pantanal.
		
		\subsubsection{O Pantanal Mato-Grossense e a Teoria de Refúgios e Redutos}
	
			As informações obtidas referente a formação do Pantanal, demorou milhares de anos para serem coletadas, passou por processos de pesquisas e análises para obter  uma resposta que esclareça sobre a origem e a evolução do Pantanal.
	
			O espaço fisiográfico do Pantanal, iniciou-se devido a uma reativação tectônica, no qual causou o rebaixamento das planícies, causando um preenchimento detrítico de uma bacia de sedimentação,  ao longo do tempo, com o desenvolvimento e adaptação da vasta abóbada regional e de terrenos antigos, consequências da erosão, originou-se a depressão pantaneira \cite{ab2006brasil}.
	        
	
		\subsubsection{Primeiros Pesquisadores}
	
			\begin{citacao}
				“As primeiras informações sobre vegetação do Pantanal foram de viagens exploratórias de alguns dos botânicos europeus pioneiros que começaram a desvendar a flora brasileira entre 1825 e 1895, por exemplo, Langsdorff, Riedel, Weddell, Kuntze, Lindman, Malme e Spencer Moore”\cite{pott2009vegetaccao}
			\end{citacao}
		
		
			O Pantanal por possuir diversas fontes de pesquisas, despertou o interesse em alguns pesquisadores europeus.
		
			De acordo com Salis, Suzana Maria et al. (2006) a Flora Pantaneira teve suas primeiras informações registrada no final do século XIX por naturalistas europeus, como o Moore (1895) e Malme (1905), que trabalharam com espécies de algumas famílias (Leguminosae e Vochysiaceae), Veloso (1947) que classificou alguns tipos de vegetações relacionando-as com regiões inundadas e a sua composição de espécies, o levantamento e listagens de algumas espécies ocorrentes no Pantanal foram realizadas por Pott et al. (1986, 1997) e Guarim Neto, a composição florística dos capões localizados no Região do Abobral, no município de Corumbá, estado de Mato Grosso do Sul, foram estudadas por Damasceno Júnior et al (1999).
		
			Prance \& Schaller (1982), Ratter et al. (1988), Nascimento \& Cunha (1989) e Cunha (1990) realizaram os trabalhos pioneiros referente aos aspectos fitossociológicos nas regiões de florestas e cerradões do Pantanal. Os estudos das estruturas em cerradões e florestas no Pantanal da Nhecolândia no estado de Mato Grosso do Sul foram realizados por Soares (1997), Salis et al. (1999) e Salis (2000).
		
			Estudos realizados por Salis, Suzana Maria et al. (2006) sobre o Cerrado brasileiro, baseando-se no trabalho de Ratter et al. (1996, 2003) e Castro et al. (1999), com o intuito de analisar os aspectos e distribuições das vegetações encontradas no Pantanal, diante de observações notou-se um padrão geográfico na distribuição da flora, no qual seis grupos florísticos foram reconhecidos após a comparação e análise de 376 áreas do Cerrado, o grupo do Sudeste, (cerrados do Estado de São Paulo, Paraná e Sul de Minas Gerai); Centro-Sudoeste (cerrados do Distrito Federal e de Minas Gerais); Norte-Nordeste (Bahia, Ceará, Norte de Minas Gerais, Maranhão, Piauí e parte do Leste de Tocantins); Centro-Oeste (Mato Grosso, Mato Grosso do Sul, Goiás, parte do Oeste de Tocantins e Sul do Paraná); mesotrófico-oeste(cerrados de vários estados, principalmente de Rondônia e Mato Grosso do Sul) e um grupo de áreas disjuntas, formado pelas savanas amazônicas.
		
	
\section{Vegetações }

	\subsection{Famílias}
	
		\subsubsection{Espécies}
		
			
		


\section{Sistemas Operacionais}
		
	De acordo com \citeonline []{toledo2016desenvolvimento}, "São chamados smartphones os telefones celulares que oferecem alta capacidade de processamento", ou seja, oferece recursos avançados e segundo \citeonline []{torres2018biblia}, é um computador que pode ser transportado a palma da mão, e como todo computador os smartphones necessitam de sistemas operacionais para funcionar.
		
	O sistema operacional é responsável por gerenciar processos e realizar a comunicação entre aplicação e hardware, além de disponibilizar ao usuário uma interface no computador, de forma que a utilização fique mais agradável. \citeonline {velloso2014informatica}
			
	Hoje em dia o mercado de dispositivos móveis oferece numerosos modelos de aparelhos, e cada fabricante opta por um determinado sistema operacional. Para este trabalho foram selecionados dois sistemas operacionais mais utilizados no mundo segundo IDC (\textbf{ver como se referencia}), o sistema operacional Android da empresa Google e o iOS da Apple.
		
	\subsection{Android}
			
		Liderado pela Google, o Android é uma Linguagem Open Source, ou seja, de código livre para dispositivos móveis, como é um projeto Open Source a documentação é pública e livre para todos, o objetivo é evitar qualquer falha no sistema através de feedbacks transmitidos por seus usuários. Dessa maneira o Android torna-se um sistema operacional completo com código-fonte personalizável que pode ser enviado para qualquer dispositivo.
				
		%https://source.android.com/
				
	\subsection{IOS}
			
				
			
\section{Desenvolvimento Mobile}

	Este capítulo tem por objetivo apresentar as categorias de desenvolvimento mobile segundo os pesquisadores \citeonline {da2014paradigmas}, \citeonline{emdesafios} e \citeonline{toledo2016desenvolvimento}.

	\begin{citacao}
		“A evolução da tecnologia dos aparelhos celulares permitiu oferecer ao usuário recursos que vão muito além da realização de uma chamada ou do envio de uma mensagem. As melhorias de hardware dos aparelhos celulares permitiram o desenvolvimento de sistemas operacionais mais avançados.”\cite[]{da2014paradigmas}
	\end{citacao}
	
	Com o avanço dos Sistemas operacionais tornou-se factível produzir aplicativos melhores, tornando os aparelhos celulares mais comuns no cotidiano das pessoas devido a comodidade em facilitar a realização de tarefas diárias.
	
	Segundo \citeonline []{emdesafios}, os aplicativos móveis possibilitaram a criação de novos modelos de negócio, em que o usuário pode ter acesso a novos recursos e serviços, por meio de lojas virtuais produzidas por fabricantes de dispositivos móveis. O principal benefício de surgimento de novos modelos é a ligação de serviços e aplicações avançadas de Internet ao aparelho mobile. Entretanto, existem diferentes plataformas tecnológicas para desenvolvimento mobile, incluindo sistemas operacionais e IDE's  (Integrated Development Environment), gerando uma ampla diversidade de soluções disponíveis no mercado. Os aplicativos móveis dependem de uma organização de código específico para que possam ser continuamente executados em diversas plataformas e versões.
	
	\begin{citacao}
		"Limitações de plataforma para distribuição do aplicativo, como, tempo, custo para o desenvolvimento e complexidade das tecnologias necessárias para a sua criação e manutenção são pontos problemáticos em um projeto voltado ao desenvolvimento deste tipo de aplicativo."\cite{emdesafios}
	\end{citacao}
	
	
	Seguindo essa linha de pensamento existem duas maneiras de categorizar aplicativos, nativas e web, ambos com seus prós e contras. Há também aplicações multiplataformas que englobam algumas funcionalidades de ambas categorias. Pensando nisso é fundamental saber qual utilizar na hora de planejar um desenvolvimento mobile.
	
	\subsection{Desenvolvimento de Aplicações Nativas}
	
		Aplicativos Nativos são aplicações desenvolvidas para determinados SOs, com Linguagens de programação e IDEs específicas. Cada qual com sua responsabilidade, O sistema operacional é responsável por gerenciar os recursos do dispositivo, a linguagem é utilizada na programação do software e o IDE dispõe ferramentas que ajudam na programação da aplicação.
		
		Com a finalidade de melhorar a interação com o usuário, aplicações nativas utilizam recursos presentes no aparelho,como dispositivos de hardware ()GPS, microfone, magnetômetro e câmera) e software (e-mails,  calendário, contatos, fotografias e telefone), para isto os aplicativos são instalados diretamente no sistema operacional, facilitando o uso da aplicação no modo off-line, caso não haja acesso a internet.\cite{toledo2016desenvolvimento}
		
		Apesar das múltiplas funcionalidades, aplicativos nativos não são fáceis de desenvolver, pois para ser executado em diversos dispositivos é preciso utilizar uma variedade de plataformas e programar de forma que funcione diversas versões. Devido a grande quantidade de plataformas, mantê-lo atualizado é desvantajoso, pois faz-se necessário uma variedade de teste e distribuição para cada versão.
	
	\subsection{Desenvolvimento de Aplicações Web}
	
		Uma aplicação web móvel nada mais é que um site em formato para celulares, que pode ser acessado de um navegador Web instalado no dispositivo. Tal qual um ambiente web tradicional, é desenvolvido utilizando as linguagens HTML para texto estáticos e imagens, CSS para o design da tela e JavaScript que é responsável pelos efeitos na aplicação. Por serem baseados em navegadores web, as aplicações podem ser excutadas em quaisquer dispositivos que possuem acesso a internet.\cite{emdesafios}
		
		Segundo \citeonline {emdesafios}, as aplicações web são mais baratas, de fácil manutenção e não ocupam memória no dispositivo,pois, não precisam ser instaladas, o que as torna vantajosas devido ao fato de serem compatíveis com diversas plataformas.
		
		Mesmo com as diversas vantagens, é necessário o acesso constante à internet em uma velocidade satisfatória para que haja uma rápida troca de dados com os servidores, além de que se o sistema precisar utilizar recursos e funcionalidades do dispositivo as aplicações web podem ser consideradas descartadas.\cite{toledo2016desenvolvimento}.
		
		Como as aplicações nativas, aplicativos web necessitam de vários testes para adequar ao tamanho de tela dos dispositivos e as exigências de cada navegador, uma vez que o mercado de smartphones é amplo e cada fabricante faz uso de navegadores que se adaptam melhor ao seus produtos.
			
	\subsection{Desenvolvimento de Aplicações Híbridas}
	
		As  aplicações híbridas é a junção de aplicativos nativos e WEB, pois como os nativos, aplicações híbridas são baixadas diretamente no dispositivo e fazer uso de funcionalidades e recursos do mesmo, por exemplo a câmera, bluetooth e GPS. Assim com aplicativos WEB, os híbridos fazem uso de linguagens HTML, JavaScript e CSS. \cite{tavares2016introduccao}
		
		A vantagem deste tipo de aplicação é a facilidade no desenvolvimento, pois é preciso apenas um código para ser compilado para diversas plataformas, poupando tempo para o  desenvolvedor, dado que não se faz necessário a programação em várias linguagens.
	
\section{Frameworks}


	\subsection{Laravel}
	
		
	
\section{Ambientes de Desenvolvimento}
	
	\subsection{Visual Studio Code}
	
	IDEs: vscode, mysql workbench
	
	Linguagens: PHP, Javascript, HTML5, CSS3
	
	Frameworks: Laravel e React Native

	\subsection{composer}
	Gerenciador de dependências do PHP.
	
	\subsection{node e npm}
	
		\subsection{JavaScript}
	
	\subsection{React Native} %citar 
	\subsection{node Js} 
 











% ----------------------------------------------------------
% PARTE
% ----------------------------------------------------------
\part{Resultados}
% ----------------------------------------------------------

% ---
% primeiro capitulo de Resultados
% ---
\chapter{Lectus lobortis condimentum}
% ---

% ---
\section{Vestibulum ante ipsum primis in faucibus orci luctus et ultrices
posuere cubilia Curae}
% ---



% ---
% segundo capitulo de Resultados
% ---
\chapter{Nam sed tellus sit amet lectus urna ullamcorper tristique interdum
elementum}
% ---

% ---
\section{Pellentesque sit amet pede ac sem eleifend consectetuer}
% ---



% ----------------------------------------------------------
% Finaliza a parte no bookmark do PDF
% para que se inicie o bookmark na raiz
% e adiciona espaço de parte no Sumário
% ----------------------------------------------------------
\phantompart

\include{elementos/textuais/conclusao}

% ----------------------------------------------------------
% ELEMENTOS PÓS-TEXTUAIS
% ----------------------------------------------------------
\postextual
% ----------------------------------------------------------

% ----------------------------------------------------------
% Referências bibliográficas
% ----------------------------------------------------------
\bibliography{elementos/referencias_bibliograficas}

% ----------------------------------------------------------
% Glossário
% ----------------------------------------------------------
%
% Consulte o manual da classe abntex2 para orientações sobre o glossário.
%
%\glossary

% ----------------------------------------------------------
% Apêndices
% ----------------------------------------------------------

% ---
% Inicia os apêndices
% ---
\begin{apendicesenv}

% Imprime uma página indicando o início dos apêndices
\partapendices

% ----------------------------------------------------------
\chapter{Quisque libero justo}
% ----------------------------------------------------------

\lipsum[50]

% ----------------------------------------------------------
\chapter{Nullam elementum urna vel imperdiet sodales elit ipsum pharetra ligula
ac pretium ante justo a nulla curabitur tristique arcu eu metus}
% ----------------------------------------------------------
\lipsum[55-57]

\end{apendicesenv}
% ---


% ----------------------------------------------------------
% Anexos
% ----------------------------------------------------------

% ---
% Inicia os anexos
% ---
\begin{anexosenv}

% Imprime uma página indicando o início dos anexos
\partanexos

% ---
\chapter{Tabela de Plantas Escolhidas}
% ---

	\begin{table}[h]
		
		\caption{Relação de Plantas}
		\begin{tabular}{ccccc}
			\hline
			Nome		  	 & Nome Científico 				   			& Família 		& Época de 			 & Grau de			\\
			Popular			 & 											&				& Floração			& Ameaça			\\
			% Note a separação de col. e a quebra de linhas
			\hline                               % para uma linha horizontal
			Acuri   		 & \textit{Attalea phalerata}	   			& Arecaceae 	& ago-jan 			& LC				\\
							 & Mart. ex Spreng				   			&				&				    &					\\
							
			Amburana		 & \textit{Amburana cearensis} 	   			& Fabaceae 		& mar-jun 			& NT				\\
							 & (Allemão) A.C.Sm				   			&				&				    &					\\
			
			Angico 			 & \textit{Albizia niopoides} 	   			& Fabaceae 		& out-jan 			& LC 				\\
							 & (Spruce ex Benth.) Burkart 	   			&				&					&					\\
								
			Aroeira 		 & \textit{Myracrodruon urundeuva} 			& Anacardiaceae	& ago-set 			& LC 				\\
							 &  Allemão									&				&				    &					\\
			
			Bocaiúva 		 & \textit{Acrocomia aculeata}      		& Arecaceae 	& out-fev 			& NE 				\\
							 &  (Jacq) Lodd. ex Mart.		    		&				&				    &					\\
			
			Cajá 			 & \textit{Spondias mombin}		    		& Anacardiaceae & out-nov 			& NE 				\\
							 & L. 						  	    		&				&				    &					\\
			
			Caju 			 & \textit{Anacardium humile}  	    		& Anacardiaceae & ago-nov 			& LC 				\\
							 &  A.St.-Hil.					    		&				&			    	&					\\
			
			Caqui do Cerrado & \textit{Diospyros hispida}  	    		& Ebenaceae 	& ago-nov 			& LC 				\\
							 &  A.DC						    		&				&				   	&					\\
			
			Caranda 		 & \textit{Copernicia alba}  	    		& Arecaceae 	& jul-dez 			& NE 				\\
							 & Morong ex Morong \& Britton	    		&				&					&					\\
			
			Gonçalo 		 & \textit{Astronium fraxinifolium}			& Anacardiaceae & jul-set 			& LC 				\\
							 &  Schott.									&				&				    &					\\
			
			Ipê-Amarelo		 & \textit{Tabebuia aurea}   				& Bignoniaceae 	& ago-out 			& NE 				\\
							 & (Silva Manso) Benth. \&	Hook.f.			&				&				    &					\\
							 & ex S.Moore								&				&				    &					\\
						
			Ipê-Branco 		 & \textit{Tabebuia roseoalba}  			& Bignoniaceae 	& ago-dez			& NE 				\\
							 & (Ridl.) Sandwith 						&				&				    &					\\
			
			Ipê-Verde 		 & \textit{Cybistax antisyphilitica}		& Bignoniaceae  & ago-mar 			& NE 				\\
							 & (Mart) Mart 								&				&				    &					\\
			
			Jenipapo 		 & \textit{Genipa americana}  				& Rubiaceae 	& set-dez 			& LC 				\\
							 & L. 										&				&				    &					\\
			
			Louro Preto 	 & \textit{Cordia glabrata}  				& Boraginaceae 	& ago-out 			& NE 				\\
							 & (Mart.) A.DC. 							&				&				    &					\\
			
			Passarinho 		 & \textit{Pterogyne nitens}  				& Fabaceae 		& fev-ago 			& LC 				\\
							 & Tul										&				&				    &					\\
			
			Pata de vaca 	 & \textit{Bauhinia rufa}	  				& Fabaceae 		& mai-nov 			& NE 				\\
							 &  (Bong.) Steud.							&				&				    &					\\
			
			Sucupira 		 & \textit{Bowdichia virgilioides}		  	& Fabaceae 		& jun-ago 			& NT 				\\
							 & Kunth									&				&				    &					\\
			
			Tarumã           & \textit{Vitex cymosa}	 				& Lamiaceae 	& set-nov 			& NE 				\\
							 &  Bertero ex Spreng.						& 				&					&					\\
			
			Ximbuva 		 & \textit{Enterolobium contortisiliquum}	& Fabaceae 		& set-nov 			& NE 				\\
							 &  (Vell.) Morong 							&				&					&					\\
			\hline
		\end{tabular}
	\end{table}

% ---
\chapter{Anexo B}
% ---



% ---
\chapter{Anexo C}
% ---



\end{anexosenv}

%---------------------------------------------------------------------
% INDICE REMISSIVO
%---------------------------------------------------------------------
\phantompart
\printindex
%---------------------------------------------------------------------

\end{document}
